% !TeX document-id = {ca41679b-2721-473b-bbe9-aab1919c455e}
% --------------------------------------------------------------------
%
%              Dissertation von Lingrui Zhu
%  Titel:      Machine Learning aided Acquistion, Feedback and Utilization of Channel State Information for Wireless Communication
%
%
% --------------------------------------------------------------------
%
%
%
%%%%%%%%%%%%%%%%%%%%%%%%%%%%%%%%%%%%%%%%%%%%%%%%%%%%%%%
%
% Toplevel-File  : dissertation.tex
%
%%%%%%%%%%%%%%%%%%%%%%%%%%%%%%%%%%%%%%%%%%%%%%%%%%%%%%%
%
%\pdfmajorversion=2
\pdfminorversion=7  % sets pdf version to 1.7, on which PDF/A-2 and PDF/A-3 are based
% Silence these warnings - intentionally
\RequirePackage{silence}
\WarningFilter{latex}{Writing or overwriting file}
\WarningFilter{glossaries}{Deprecated command}
% Enable PDF/A meta data stuff. Needs to be here at the top.
\begin{filecontents*}[overwrite]{\jobname.xmpdata}
    \Title{The title meta data that one sees in the PDFA}
    \Author{Author name}
    \Language{en-US}
    \Subject{Preferably your thesis abstract}
    \Keywords{5G\sep URLLC\sep Industrie 4.0\sep Industrial Radio\sep SDR\sep Software\sep Reliability\sep Latency}
\end{filecontents*}

\documentclass[10pt,a5paper,twoside,pdftex,dvipsnames]{book}
%
% Define base commands for title page and PDF properties
%
\newcommand{\Arbeitsart}{Dissertation \\ zur Erlangung des akademischen Grades \\
            {\em Doktor der Ingenieurwissenschaften (Dr.-Ing.)}}
\newcommand{\Autor}{M. Sc. Lingrui Zhu}
\newcommand{\Berichtstitel}{Machine Learning Aided Acquistion, Feedback and Utilization of Channel State Information for Wireless Communication}
%
% Load packages -> this order worked, change at own risk!
%
\usepackage[english]{babel}               % Language selection
\usepackage[utf8]{inputenc}               % Ensure correct input handling
\usepackage[T1]{fontenc}                  % Specifies font encoding
\usepackage{lmodern}                      % Modern latin vector fonts
\usepackage{amsmath,amsfonts,amssymb,bm}  % Typical math packages
\usepackage{ushort}                       % Short underline, e.g., to use with y
\usepackage{fancyhdr}                     % Advanced header formatting
\usepackage{float}                        % Enhanced float control
\usepackage[chapter]{algorithm}           %
\usepackage{algorithmic}                  % Algorithm environments and floats
\usepackage{eqparbox}                     % Fixed size box for algorithm tweaking
\usepackage{siunitx}                      % Units definition, also for numbers
\DeclareSIUnit{\sample}{S}                % Example definition
\usepackage{makecell}                     % Common layouts for tabular column heads, multi-lined tabular cells
\usepackage{listings}                     % Typeset programming code, e.g., Python
\usepackage{dirtytalk}                    % Easy quotes with /say{}
\usepackage{scrextend}                    % Page layout based on established typographic standards
%\usepackage{layouts}                     % Display information of a document's layout
\usepackage[normalem]{ulem}               % Improved underlining
\usepackage[table]{xcolor}                % Enable colored cells in tables

%% PDF/A compliance (see https://webpages.tuni.fi/latex/pdfa-guide.pdf)
% Note: Check compliance with veraPDF in the end
\usepackage{colorprofiles}
\usepackage[a-2b,mathxmp]{pdfx}[2018/12/22] % includes hyperref and xcolor among other packages
% Options such as : `pdfpagelabels,backref=page` do not work with PDF/A
% Commented out data is already included in PDF/A preamble. No need to to it twice. Will only break conformance.
\hypersetup{
                 % pdftitle={blah blah},
                 % pdfauthor={\Autor},
                 % pdflang={en-US},
                 % pdfcreator = {pdfLaTeX with hyperref},
                 % pdfsubject={Doktorarbeit an der Uni Bremen},
                 % pdfkeywords = {MIMO, Adaptivity, OFDM, Multiuser},
                 colorlinks=false, % links have no specific color -> may be bad for online versions
                 linkcolor=red,
                 anchorcolor=black,
                 citecolor=red,
                 filecolor=magenta,
                 menucolor=red,
                 urlcolor=cyan,
                 urlbordercolor=0 0 0,
                 pdfborder=0 0 0,
                 pdffitwindow=true,
                 pdfstartpage = 1,
                 pdfstartview=Fit,
                 pdfpagelayout=TwoColumnRight,
                 pdfpagemode=UseOutlines,
                 bookmarksopen = true,
                 bookmarksnumbered = true
                 }

%
\usepackage{graphicx}                 % Advanced graphics inclusion
\graphicspath{{images/}{images/demo}{schematics/}{schematics/implementation}} % Set the graphicspath
\usepackage{svg}                      % SVG support, requires Inkscape
\svgpath{{images/}}                   % Set svg path
\usepackage{tcolorbox}                % Colored and frame text boxes
\usepackage{tikz}                     % Tikz and PGFplots for cool pics :)
\usepackage{pgfplots}
% https://tex.stackexchange.com/questions/7953/how-to-expand-texs-main-memory-size-pgfplots-memory-overload
\usetikzlibrary{pgfplots.external}    % Externalize tikz-pictures to pdfs
\tikzexternalize[prefix=TikZexternal/]% Externalization path

\usepackage[hang,small,bf]{caption}   % Advanced Caption formatting
\usepackage{subcaption}
% Glossary package for index, acronyms and symbol list -> <=v4.49 necessary for compatibility, reprogramming neccesary for >=v4.50 !!!
% Run pdflatex -> biber -> makeglossaries -> 2 * pdflatex for the glossaries to be included, makeglossaries requires perl
\usepackage[toc, acronym, shortcuts, nomain]{glossaries}[=v4.49]

\usepackage{todonotes}                % Nifty todo notes
\usepackage{longtable, booktabs}      % Tabellen ueber mehr als eine Seite
\usepackage{multirow}                 % Text in multiple rows
\usepackage{rotfloat}                 % Rotate floats
\usepackage[babel]{microtype}         % Enhanced Type-Setting

% Advanced bibliographies
% !BIB TS-program = biber
\usepackage[backend=biber, style=ieee-alphabetic, maxbibnames=99, maxcitenames = 10, doi=true, url=false, isbn=false, giveninits = true, dateabbrev=true, abbreviate=true, useprefix=true]{biblatex}
\usepackage{csquotes}
%\usepackage{showkeys}                % Look for duplicate entries
\AtBeginBibliography{\small}          % Small fontsize for references
\defbibheading{subbibliography}{      % Make heading lowercase
  \section*{#1}%
  \addcontentsline{toc}{section}{#1}%
  \markboth{#1}{#1}%
}
\renewcommand*{\labelalphaothers}{\textsuperscript{+}} % Change +-sign to thesis default
% Add bibliographies here
\addbibresource{include/IEEEfull.bib} % add IEEE journal names/abbreviations
%\addbibresource{include/IEEEabrv.bib}
\addbibresource{include/bibliography.bib}
% Clear "visited on" for all entries
\AtEveryBibitem{
    \clearfield{urlyear}
    \clearfield{urlmonth}
}
% IEEE style fix for full month names and to fix volume abbreviation according to PhD thesis convention
%\DefineBibliographyStrings{english}{
  %june      = June,
  %july      = July,
  %september = September, % orig english.lbx has Sept\adddot; IEEE is Sep\adddot
  %jourvol = volume,
%}


% Fine tuning the algorithm style
\algsetup{indent=2em}

% Ensure equally indented comments dynamically fit to the longest comment
\renewcommand{\algorithmiccomment}[1]{\hfill\eqparbox{COMMENT}{\# #1}}

%
%%%%%%%%%%%%%% Shorthands and math definitions%%%%%%%%%%%%%%%
%%%%%%%%%%%%%%%%%%%%%%%%%%%%%%%%%%%%%%%%%%%%%%%%%%%%%%%%%%%%%%%%%%%%%%%%%%%%%%%
% Macro definitions for mathematical symbols and functions                    %
%%%%%%%%%%%%%%%%%%%%%%%%%%%%%%%%%%%%%%%%%%%%%%%%%%%%%%%%%%%%%%%%%%%%%%%%%%%%%%%

% General Stuff
\newcommand{\D}{\displaystyle}
\newcommand{\T}{\textstyle}

% C++ typesetting
\newcommand{\Cpp}{C\nolinebreak\hspace{-.05em}\raisebox{.4ex}{\tiny\bf +}\nolinebreak\hspace{-.10em}\raisebox{.4ex}{\tiny\bf +}}
\newcommand{\Arikan}{Ar{\i}kan}

% Reference shortcuts
\newcommand{\reffig}[1]{Fig.~\ref{#1}}
\newcommand{\reftable}[1]{Tab.~\ref{#1}}
\newcommand{\refapp}[1]{Appendix~\ref{#1}}
\newcommand{\refchap}[1]{Chapter~\ref{#1}}
\newcommand{\refsec}[1]{Sec.~\ref{#1}}

% Symbol modifiers
\newcommand{\softValue}[1]{\tilde{#1}}
\newcommand{\hardValue}[1]{\hat{#1}}

% Indices
\newcommand{\idxC}{\ensuremath{k}}      % Carrier index
\newcommand{\idxT}{\ensuremath{t}}      % Continuous time index
\newcommand{\idxTau}{\ensuremath{\tau}} % Some continuous time
\newcommand{\idxN}{\ensuremath{n}}      % Discrete time index
\newcommand{\idxL}{\ensuremath{\ell}}   % channel length index
\newcommand{\idxA}{\ensuremath{\nu}}    % Layer index

\newcommand{\idxSa}{\ensuremath{\iota}}  % Arbitrary Index
\newcommand{\idxSb}{\ensuremath{\kappa}} % Arbitrary Index
\newcommand{\idxSc}{\ensuremath{\rho}}   % Arbitrary Index

% Bracket Definitions
\newcommand{\lpa}{\ensuremath\left(}
\newcommand{\rpa}{\ensuremath\right)}
\newcommand{\lbb}{\ensuremath\left[}
\newcommand{\rbb}{\ensuremath\right]}
\newcommand{\lcb}{\ensuremath\left\{}
\newcommand{\rcb}{\ensuremath\right\}}

% Misc. definitions
\newcommand{\vala}{\ensuremath \zeta} % Arbitrary value / variable
\newcommand{\valb}{\ensuremath \upsilon} % Another variable
\newcommand{\const}{\ensuremath c} % A constant value
\newcommand{\mI}[1][]{\ensuremath \mathbf{I}_{#1}} % identitiy matrix
\newcommand{\mO}[1][]{\ensuremath \mathbf{0}_{#1}} % zero matrix}
\newcommand{\no}{\ensuremath N} % Some number of terms / elements, etc...
\newcommand{\func}[2][]{\ensuremath f_{#1}\negthinspace\lpa#2\rpa} % Arbitrary function
\newcommand{\besselI}[1][\cdot]{\ensuremath I_0\negthinspace\lpa#1\rpa} % Modified Bessel function first kind, zeroth order
\newcommand{\noce}[1][\idxC]{\ensuremath \lambda_{#1}} %Non-Centrality of the chi^2-distribution

\newcommand{\mF}[1][\nC]{\ensuremath \mathbf{F}_{\nC}} % FFT Matrix
\newcommand{\su}{\ensuremath{u}} % Arbitrary element
\newcommand{\vu}{\ensuremath \mathbf{\su}} %Arbitrary vector

% Optimization
\DeclareMathOperator*{\argmax}{argmax}
\DeclareMathOperator*{\argmin}{argmin}
\newcommand{\lvar}{\ensuremath{\lambda}}
\newcommand{\lagrf}{\ensuremath{J}}
\newcommand{\lagr}[1][\lvar]{\ensuremath \lagrf\negthinspace\lpa#1\rpa}
\newcommand{\lkup}[1][\zeta]{\ensuremath L\negthinspace\lpa#1\rpa}

% Statistics
\DeclareMathOperator{\p}{p}
\DeclareMathOperator{\pp}{P}
\DeclareMathOperator{\e}{e}
\DeclareMathOperator{\E}{E}
\newcommand{\Prob}[1]{\ensuremath  \pp\negthinspace\lpa#1\rpa} % probability
\newcommand{\cProb}[2]{\ensuremath  \pp\negthinspace\lpa#1\vert#2\rpa} % conditional probability

\newcommand{\pdf}[2][]{\ensuremath \p_{#1}\negthinspace\lpa#2\rpa} % probability density function
\newcommand{\cpdf}[3][]{\ensuremath \p_{#1}\negthinspace\lpa#2\vert#3\rpa} % conditioned pdf
\newcommand{\cdf}[2][]{\ensuremath \pp_{#1}\negthinspace\lpa#2\rpa} % cumulative density Function

\newcommand{\epc}[2][]{\ensuremath \E_{#1}\negthinspace\lcb#2\rcb} % Expectation of a random variable
\newcommand{\std}[1][]{\ensuremath \sigma_{#1}} % Standard deviation 
\newcommand{\mean}[1][]{\ensuremath \mu_{#1}} % Mean

\newcommand{\CFun}[1][]{\ensuremath \Phi_{#1}\!\lpa\omega\rpa}

\newcommand{\dprob}[1][\text{out}]{\ensuremath \pp_{#1}} 

\newcommand{\corr}{\ensuremath r} % Correlation

% Prob. distributions
\newcommand{\rnormpdf}[1][0,1]{\ensuremath\mathcal{N}\negthinspace\lpa#1\rpa}
\newcommand{\cnormpdf}[1][0,1]{\ensuremath\mathcal{N}_{\text{C}}\negthinspace\lpa#1\rpa}
\newcommand{\ierf}[1][\cdot]{\ensuremath\mathrm{erf}^{-1}\lpa#1\rpa}
\newcommand{\erf}[1][\cdot]{\ensuremath\mathrm{erf}\lpa#1\rpa}
\newcommand{\erfc}[1][\cdot]{\ensuremath\mathrm{erfc}\lpa#1\rpa}
\newcommand{\Qtail}[1][\cdot]{\ensuremath Q\negthinspace\lpa#1\rpa}
\newcommand{\Qtailinv}[1][\cdot]{\ensuremath Q^{-1}\negthinspace\lpa#1\rpa}

% Approximation
\newcommand{\aProx}[1][]{\ensuremath \mathfrak{a}_{#1}}
\newcommand{\bProx}[1][]{\ensuremath \mathfrak{b}_{#1}}

% Operators
\DeclareMathOperator{\dg}{diag}
\DeclareMathOperator{\lowtri}{tril}
\DeclareMathOperator{\vect}{vec}
\DeclareMathOperator{\trace}{Tr}
\DeclareMathOperator{\md}{mod}
\newcommand{\diag}[1][\cdot]{\ensuremath\dg\lpa#1\rpa}
\newcommand{\tril}[1][\cdot]{\ensuremath\lowtri\lpa#1\rpa}
\newcommand{\abs}[1][\cdot]{\ensuremath\left\lvert#1\right\rvert}
\newcommand{\norm}[1][\cdot]{\ensuremath\left\lVert#1\right\rVert}
\newcommand{\Tr}[1][\cdot]{\ensuremath \trace\lcb#1\rcb}
\newcommand{\re}[1][\cdot]{\ensuremath \mathfrak{Re}\negthinspace\lcb#1\rcb} % real part
\newcommand{\im}[1][\cdot]{\ensuremath \mathfrak{Im}\negthinspace\lcb#1\rcb} % imaginary part
\newcommand{\sign}[1][\cdot]{\ensuremath \mathrm{sgn}\lcb#1\rcb} % Sign of a scalar

% Sets 
\newcommand{\eA}{\ensuremath{a}} % Element of set A
\newcommand{\setA}[1][]{\ensuremath\mathcal{\MakeUppercase{\eA}}_{#1}} % Set of all possible symbol alphabet elements
\newcommand{\defA}[1][]{\ensuremath\lcb\eA_1,\dotsc,\eA_{\M_{#1}}\rcb} % Definition of set A

\newcommand{\fmap}[2][]{\ensuremath\mathcal{M}_{#1}\negthinspace\lpa#2\rpa} % Mapping function
\newcommand{\deffmap}[1][]{\ensuremath\mathcal{M}_{#1}:\mathrm{GF}\negthinspace\lpa2\rpa^{\ldM[#1]\times 1}\mapsto\setA[#1]}

\newcommand{\setD}[1][]{\ensuremath{\mathcal{D}_{#1}}} % N-dimensional set of symbol vectors
\newcommand{\defD}[1][0]{\ensuremath\lcb\vdwir[\idxC]\,\vert\,\sd_{\idxC,\idxA}\in\ifthenelse{#1>0}{\setA[\idxC,\idxA]}{\setA}\quad\forall\,\idxA=1,\dotsc,2\nA\rcb} % Definition of set D
\newcommand{\defdmap}[1][]{\ensuremath\mathcal{M}_{#1}:\mathrm{GF}\negthinspace\lpa2\rpa^{\nd[#1]\times 1}\mapsto\setD[#1]}

\newcommand{\setU}{\ensuremath\mathcal{U}} % Set of active Users

\newcommand{\setPi}{\ensuremath\mathcal{I}} % Set of indexes (e.g., for interleavers)
\newcommand{\setS}[1][]{\ensuremath\mathcal{S}_{#1}} % Set of subcarriers assigned to one user

\newcommand{\setR}{\ensuremath\mathcal{R}_{\text{C}}} % Set of applicable code rates 
\newcommand{\setL}{\ensuremath\mathcal{L}} % Set of rate-power points

\newcommand{\domr}{\ensuremath\mathbb{R}} % real domain
\newcommand{\domc}{\ensuremath\mathbb{C}} % complex domain

% Information theory
\newcommand{\capa}{\ensuremath C} % Channel capacity
\newcommand{\acap}{\ensuremath \hat{C}} % Channel capacity approximation
\newcommand{\gap}[1][]{\ensuremath \gamma_{#1}} % Capacity gap for M-QAM modulations
\newcommand{\ety}[1][x]{\ensuremath h\negthinspace\lpa#1\rpa} % differential Entropy of a signal
\newcommand{\lety}[1][x]{\ensuremath \tilde{h}\negthinspace\lpa#1\rpa} % Lower bound of the entropy of a signal
\newcommand{\eety}[1][x]{\ensuremath \hat{h}\negthinspace\lpa#1\rpa} % Approximate entropy of a signal
\newcommand{\emi}{\ensuremath I} % Extrinsic Mutual information

% Rates and Users
\newcommand{\rate}{\ensuremath R} % A rate
\newcommand{\ru}[1][]{\ensuremath r_{#1}}
\newcommand{\vru}{\ensuremath \mathbf{r}}
\newcommand{\srate}{\ensuremath \rate_{\text{ov}}}

% Error rates
\newcommand{\fer}[1][f]{\ensuremath \pp_{\text{#1}}\negthinspace}
\newcommand{\ber}[1][b]{\ensuremath \pp_{\text{#1}}\negthinspace}
\newcommand{\ser}[1][s]{\ensuremath \pp_{\text{#1}}\negthinspace}

% System Model
%% Constants
\newcommand{\nT}{\ensuremath N_{\text{T}}} % Number of transmit antennas
\newcommand{\nA}{\ensuremath N_{\text{L}}} % Number of layers
\newcommand{\nTst}{\ensuremath N_{\text{ST}}} % Number of time slots for STP
\newcommand{\nR}{\ensuremath N_{\text{R}}} % Number of receive antennas
\newcommand{\nU}{\ensuremath N_{\text{U}}} % Number of users
\newcommand{\nC}{\ensuremath N_{\text{C}}} % Number of carriers
\newcommand{\nCP}{\ensuremath N_{\text{CP}}} % Guard Interval Length
\newcommand{\nS}{\ensuremath N_{\text{S}}} % Length of one OFDM Symbol
\newcommand{\nG}{\ensuremath N_{\text{G}}} % Number of carriers in a chunk
\newcommand{\nL}{\ensuremath N_{\text{F}}} % Number of channel taps
\newcommand{\nCor}{\ensuremath N_{\text{Cor}}} % Number of correlated samples
\newcommand{\nCW}{\ensuremath N_{\text{W}}} % Size of the correlation window

%% Code and BICM Description
\newcommand{\rc}[1][]{\ensuremath \ifthenelse{\equal{#1}{}}{R_{\text{C}}}{R_{\text{C},#1}}} % Code rate
\newcommand{\vrc}[1][]{\ensuremath \ifthenelse{\equal{#1}{}}{\mathbf{r}_{\text{C}}}{\mathbf{r}_{\text{C},#1}}} % Code rate vector
\newcommand{\rce}[1][]{\ensuremath \ifthenelse{\equal{#1}{}}{\hat{R}_{\text{C}}}{\hat{R}_{\text{C},#1}}} % Code rate estimate

\newcommand{\symBaseband}{s}
\newcommand{\symBasebandR}{\softValue{\symBaseband}}
\newcommand{\vecBaseband}{\ensuremath \bm{\symBaseband}}
\newcommand{\vecBasebandR}{\ensuremath \bm{\symBasebandR}}

\newcommand{\symNoise}{n}
\newcommand{\vecNoise}{\ensuremath \bm{\symNoise}}

\newcommand{\symChannelTap}{h}
\newcommand{\matChannelTaps}{\ensuremath \bm{\MakeUppercase{\symChannelTap}}}

\newcommand{\drc}[1][]{\ensuremath \ifthenelse{\equal{#1}{}}{R'_{\text{C}}}{R'_{\text{C},#1}}} % Design Code rate
\newcommand{\drce}{\ensuremath \Delta_{R'_\text{C}}} % Design Code rate error

\newcommand{\lc}{\ensuremath L_{\text{C}}} % Constraint length of channel codes
\newcommand{\mr}{\ensuremath m_{\text{R}}} % memory of channel codes
\newcommand{\rp}{\ensuremath N_{\text{P}}} % Puncturing period
\newcommand{\rpun}{\ensuremath R_{\text{P}}} % Puncturing rate 
\newcommand{\rdop}{\ensuremath R_{\text{D}}} % Doping Rate
\newcommand{\rep}[1][]{\ensuremath N_{\text{rep} #1}} % Repetion factor
\newcommand{\mrep}{\ensuremath N_{\text{rep,max}}} % Maximum Repetion factor
\newcommand{\cit}{\ensuremath L_{\text{I}}} % Number of Decoder Iterations
\newcommand{\linf}{\ensuremath \alpha} % Linear combination factor
\newcommand{\tra}{\ensuremath T} % Tranfer characteristic
\newcommand{\iradis}{\ensuremath \Delta_{\tra}} % Distance of transfer characteristics

\newcommand{\rcgap}{\ensuremath \epsilon} % Code Rate Design Gap

\newcommand{\fint}[1][\cdot]{\ensuremath \pi\negthinspace\lpa#1\rpa} % Interleaver function
\newcommand{\ifint}[1][\cdot]{\ensuremath \pi^{-1}\negthinspace\lpa#1\rpa} % Interleaver function

\newcommand{\M}{\ensuremath{M}} % Number of symbols in an alphabet
\newcommand{\vM}{\ensuremath \mathbf{\MakeLowercase{\M}}} % Vector of M for multiple layers
\newcommand{\ldM}[1][]{\ensuremath \log_2\lpa\M_{#1}\rpa} % Number of bits in a symbol

\newcommand{\bl}{\ensuremath b} % Bit label

\newcommand{\llr}{\ensuremath L} % LLR value
\newcommand{\fllr}[2][]{\llr_{#1}\negthinspace\lpa#2\rpa} % L-value shorthand
\newcommand{\fllra}[1]{\fllr[\text{A}]{#1}} % A-Priori L-value
\newcommand{\fllre}[1]{\fllr[\text{E}]{#1}} % Extrinsic L-value
\newcommand{\fllrc}[1]{\fllr[\text{C}]{#1}} % Channel L-value

%% Vector & Matrix
\newcommand{\sz}{\ensuremath{z}} % Element of signal after filtering
\newcommand{\vz}{\ensuremath\mathbf{\sz}} % Vector

\newcommand{\sW}{\ensuremath{w}} % Element of the receiver filter matrix
\newcommand{\vW}{\ensuremath\mathbf{\sW}} % Vector of the receive filter matrix
\newcommand{\mW}{\ensuremath\mathbf{\MakeUppercase{\sW}}} % Receive Filter matrix

\newcommand{\sy}{\ensuremath{y}} % Element of the receive vector
\newcommand{\vy}{\ensuremath\mathbf{\sy}} % Receive vector

\newcommand{\sSy}{\ensuremath{\Theta}} % Element of the system matrix
\newcommand{\vSy}{\ensuremath\mathbf{\sSy}} % Column of the system matrix
\newcommand{\mSy}{\ensuremath\mathbf{\MakeUppercase{\sSy}}} % System matrix 

\newcommand{\sSye}{\ensuremath{\tilde{\Theta}}} % Element of the estimated system matrix
\newcommand{\vSye}{\ensuremath\mathbf{\sSye}} % Column of the estimated system matrix
\newcommand{\mSye}{\ensuremath\mathbf{\MakeUppercase{\sSye}}} % Estimated System matrix 

\newcommand{\sH}{\ensuremath{h}} % Element of the channel matrix
\newcommand{\vH}{\ensuremath\mathbf{\sH}} % Column of the channel matrix
\newcommand{\mH}{\ensuremath\mathbf{\MakeUppercase{\sH}}} % Channel Matrix

\newcommand{\sHe}{\ensuremath{\tilde{h}}} % Element of the estimated channel matrix
\newcommand{\vHe}{\ensuremath\mathbf{\sHe}} % Column of the estimated channel matrix
\newcommand{\mHe}{\ensuremath\mathbf{\MakeUppercase{\sHe}}} % Estimated Channel Matrix

\newcommand{\sHd}{\ensuremath{\Delta}} % Element of the error channel matrix
\newcommand{\vHd}{\ensuremath\mathbf{\sHd}} % Column of the errror channel matrix
\newcommand{\mHd}{\ensuremath\mathbf{\MakeUppercase{\sHd}}} % Error Channel Matrix

\newcommand{\sG}{\ensuremath{g}} % Element of the generator matrix
\newcommand{\vG}{\ensuremath\mathbf{\sG}} % Column of the generator matrix
\newcommand{\mG}{\ensuremath\mathbf{\MakeUppercase{\sG}}} % Generator Matrix

\newcommand{\sPrec}{\ensuremath{\xi}} % Non-linear precoding vector element
\newcommand{\vPrec}{\ensuremath\boldsymbol{\sPrec}} % Non-linear precoding vector
\newcommand{\mPre}{\ensuremath\mathbf{F}} % Forward Precoding matrix
\newcommand{\mBPre}{\ensuremath\mathbf{B}} % Feedback Precoding matrix
\newcommand{\sNorm}{\ensuremath\beta} % Power normalization factor

\newcommand{\sP}{\ensuremath{p}} % Element of the power matrix
\newcommand{\vP}{\ensuremath\mathbf{\sP}} % Column of the power matrix
\newcommand{\mP}{\ensuremath\mathbf{\MakeUppercase{\sP}}} % Power Matrix

\newcommand{\sx}{\ensuremath{x}} % Element of the transmit vector
\newcommand{\vx}{\ensuremath\mathbf{\sx}} % Transmit vector

\newcommand{\sd}{\ensuremath{d}} % Element of the symbol vector
\newcommand{\vd}{\ensuremath\mathbf{\sd}} % Symbol vector
\newcommand{\mD}{\ensuremath\mathbf{\MakeUppercase{\sd}}} % Transmit Symbol matrix
\newcommand{\nd}[1][\idxC]{\ensuremath N_{\vd,#1}}

\newcommand{\sco}{\ensuremath{c}} % Element of the code bit vector
\newcommand{\vc}{\ensuremath\mathbf{c}} % Transmit vector
\newcommand{\nc}{\ensuremath N_{\vc}} % Number of code bits


%\newcommand{\sb}{\ensuremath{b}} % Element of the information bit vector
\newcommand{\vb}{\ensuremath\mathbf{b}} % Transmit vector
\newcommand{\nb}{\ensuremath N_{\vb}} % Number of information bits

\newcommand{\sbe}{\ensuremath\hat{b}} % estimated bit
\newcommand{\vbe}{\ensuremath\hat{\vb}} % estimated bit vector

\newcommand{\sn}{\ensuremath{\eta}} % Element of the noise vector
\newcommand{\vn}{\ensuremath\boldsymbol{\sn}} % Noise vector

%% Power and SNR
\newcommand{\Pow}{\ensuremath \mathcal{P}} % Total transmit power constraint
\newcommand{\ipow}{\ensuremath \MakeUppercase{\sP}} % Instantaneous transmit power
\newcommand{\mpow}{\ensuremath \overline{\MakeUppercase{\sP}}} % Mean transmit power

\newcommand{\sgn}{\ensuremath\std[\sn]} % Variance of the noise vector elements
\newcommand{\sgneff}[1][\idxC,\idxA]{\ensuremath\std[\tilde{\sn}_{#1}]} % Variance of the noise vector elements

\newcommand{\cnr}[1][\idxC]{\ensuremath\ifthenelse{\equal{#1}{}}{\gamma_{\text{CNR}}}{\gamma_{\text{CNR},#1}}} % Channel-to-noise-ratio
\newcommand{\mcnr}[1][\idxC]{\ensuremath\ifthenelse{\equal{#1}{}}{\bar{\gamma}_{\text{CNR}}}{\bar{\gamma}_{\text{CNR},#1}}} % Channel-to-noise-ratio
\newcommand{\snr}{\ensuremath{\gamma}} % Signal-to-noise-ratio
\newcommand{\msnr}{\ensuremath\overline{\snr}} % Mean Signal-to-noise-ratio
\newcommand{\tsnr}{\ensuremath\hat{\gamma}} % SNR Threshold

%% Matrix transforms
\newcommand{\mU}{\ensuremath \mathbf{U}} % Unitary matrix
\newcommand{\mS}{\ensuremath \mathbf{S}} % Matrix of singular values
\newcommand{\mV}{\ensuremath \mathbf{V}} % Unitary matrix

\newcommand{\mQ}{\ensuremath \mathbf{Q}} % Unitary Matrix
\newcommand{\mL}{\ensuremath \mathbf{L}} % Lower triangular Matrix
\newcommand{\mR}{\ensuremath \mathbf{R}} % Upper triangular Matrix

\newcommand{\sing}{\ensuremath{s}} % Singular value
\newcommand{\rankS}{\ensuremath{r}} % rank of the singular value matrix
\newcommand{\defmS}{\ensuremath \mS = \diag[\sing_1,\dotsc,\sing_{\rankS}]} % Definition of the singular value matrix

% Dimensionality
\newcommand{\cdimxt}{\ensuremath\domc^{\nT\times 1}} % Dim of the transmit vector (time domain)
\newcommand{\cdimnt}{\ensuremath\domc^{\nR\times 1}} % Dim of the noise vector (time domain)
\newcommand{\cdimyt}{\ensuremath\domc^{\nR\times 1}} % Dim of the receive vector (time domain)

\newcommand{\cdimH}{\ensuremath\domc^{\nR\times\nT}} % Dim of the channel matrix (complex domain)
\newcommand{\cdimx}{\ensuremath\domc^{\nT\times 1}} % Dim of the transmit vector (complex domain)
\newcommand{\cdimy}{\ensuremath\domc^{\nR\nTst\times 1}} % Dim of the receive vector (complex domain)
\newcommand{\cdimn}{\ensuremath\domc^{\nR\nTst\times 1}} % Dim of the noise vector (complex domain)
\newcommand{\cdimd}{\ensuremath\domc^{\nA\times 1}} % Dim of the symbol vector (complex domain)
\newcommand{\cdimp}{\ensuremath\domc^{\nT\times 1}} % Dim of the power vector (complex domain)

\newcommand{\rdimH}{\ensuremath\domr^{2\nT\times2\nR}} % Dim of the channel matrix (real domain)
\newcommand{\cdimG}{\ensuremath\domc^{\nT\nTst\times2\nA}} % Dim of the generator matrix (real domain)
\newcommand{\rdimx}{\ensuremath\domr^{2\nA\times 1}} % Dim of the transmit vector (real domain)
\newcommand{\rdimy}{\ensuremath\domr^{2\nR\times 1}} % Dim of the receive vector (real domain) 
\newcommand{\rdimn}{\ensuremath\domr^{2\nR\times 1}} % Dim of the noise vector (real domain)
\newcommand{\rdimd}{\ensuremath\domr^{2\nT\nTst\times 1}} % Dim of the symbol vector (real domain)
\newcommand{\rdimp}{\ensuremath\domr^{2\nT\times 1}} % Dim of the power vector (real domain)

\newcommand{\rdimW}{\ensuremath\domr^{2\nA\times\2\nR\nTst}} % Dim of the receive filter matrix (real domain)

\newcommand{\bdimb}{\ensuremath\mathrm{GF}\negthinspace\lpa2\rpa^{\nb\times 1}} % Dim of the power vector (real domain)
\newcommand{\bdimc}{\ensuremath\mathrm{GF}\negthinspace\lpa2\rpa^{\nc\times 1}} % Dim of the power vector (real domain)

% Derived Symbols
% continous time 
\newcommand{\vxot}{\ensuremath\underline{\vx}(\idxT)}
\newcommand{\sxot}[1][\idxSa]{\ensuremath\underline{\sx}_{#1}(\idxT)}
\newcommand{\mHot}{\ensuremath\underline{\mH}(\idxT,\idxTau)}
\newcommand{\sHot}[1][\idxSa,\idxSb]{\ensuremath\underline{\sH}_{#1}(\idxT,\idxTau)}
\newcommand{\vnot}{\ensuremath\underline{\vn}(\idxT)}
\newcommand{\vyot}[1][\idxT]{\ensuremath\ushort{\vy}\!\lpa#1\rpa}
\newcommand{\syot}[1][\idxSb]{\ensuremath\ushort{\sy}_{#1}(\idxT)}

% discrete time
\newcommand{\vxon}[1][\idxN]{\ensuremath\underline{\vx}(#1)}
\newcommand{\sxon}[1][\idxSa]{\ensuremath\underline{\sx}(\idxN,#1)}
\newcommand{\mHon}{\ensuremath\underline{\mH}(\idxN,\idxL)}
\newcommand{\vnon}{\ensuremath\underline{\vn}(\idxN)}
\newcommand{\syon}[1][\idxN]{\ensuremath\ushort{\sy}\!\lpa#1\rpa}
\newcommand{\vyon}[1][\idxN]{\ensuremath\ushort{\vy}\!\lpa#1\rpa}

% complex valued
\newcommand{\mHwi}[1][\idxC]{\ensuremath\underline{\mH}_{#1}}
\newcommand{\sHwi}[1]{\ensuremath\underline{\sH}_{#1}}
\newcommand{\vHwi}[1][\idxC]{\ensuremath\underline{\vH}_{#1}}
\newcommand{\mGwi}[1][\idxC]{\ensuremath\underline{\mG}_{#1}}
\newcommand{\mSywi}[1][\idxC]{\ensuremath\ushort{\mSy}_{#1}}

\newcommand{\mHwie}[1][\idxC]{\ensuremath\underline{\mHe}_{#1}}
\newcommand{\sHwie}[1][\idxC]{\ensuremath\underline{\sHe}_{#1}}
\newcommand{\vHwie}[1][\idxC]{\ensuremath\underline{\vHe}_{#1}}

\newcommand{\mHwid}[1][\idxC]{\ensuremath\underline{\mHd}_{#1}}
\newcommand{\sHwid}[1][\idxC]{\ensuremath\underline{\sHd}_{#1}}
\newcommand{\vHwid}[1][\idxC]{\ensuremath\underline{\vHd}_{#1}}

\newcommand{\sge}[1][]{\ensuremath \ifthenelse{\equal{#1}{}}{\std[\text{e}]}{\std[\text{e},#1]}}
\newcommand{\corre}[1][]{\ensuremath \ifthenelse{\equal{#1}{}}{\rho_{\text{e}}}{\rho_{\text{e},#1}}}

\newcommand{\mDwi}{\ensuremath\underline{\mD}} 
\newcommand{\sdwi}[1][]{\ensuremath\underline{\sd}_{#1}} 

\newcommand{\vxwi}[1][\idxC]{\ensuremath\underline{\vx}_{#1}}
\newcommand{\sxwi}[1][]{\ensuremath\underline{\sx}_{#1}}
\newcommand{\sgx}[1][]{\ensuremath\std[\underline{\sx}]}

\newcommand{\vdA}{\ensuremath\mathbf{\Delta}_{\eA_{\idxSa,\idxSb}}}
\newcommand{\defvdA}{\ensuremath\mathbf{\Delta}_{\eA_{\idxSa,\idxSb}} = \sqrt{\sP}\lpa\eA_{\idxSa} - \eA_{\idxSb}\rpa}

\newcommand{\sdxwi}[1][]{\ensuremath\mathbf{\Delta}_{\sxwi[#1]}}
\newcommand{\defsdxwi}[2][1]{\ensuremath\mathbf{\Delta}_{\sxwi[#1]} = \sqrt{\sP}\lpa\sdwi[#2] - \sdwi[#1]\rpa}

\newcommand{\vdxwi}[1][]{\ensuremath\mathbf{\Delta}_{\vxwi[#1]}}
\newcommand{\defvdxwi}[2][1]{\ensuremath\mathbf{\Delta}_{\underline{\vx}} = \mP^{1/2}\lpa\vdwi[#2] - \vdwi[#1]\rpa}
\newcommand{\setDi}{\ensuremath\mathcal{O}} % Set of all difference symbols
\newcommand{\defDi}{\ensuremath\setDi = \lcb\vdxwi = \mP^{1/2}\lpa\vdwi[\idxSa] - \vdwi[\idxSb]\rpa\,\vert\,\vdwi[\idxSa],\vdwi[\idxSb]\in\setD\,\forall\idxSa,\idxSb\rcb} % Definition of set X

\newcommand{\vywi}[1][\idxC]{\ensuremath\ushort{\vy}_{#1}}
\newcommand{\sywi}[1][]{\ensuremath\ushort{\sy}_{#1}}
\newcommand{\sgy}[1][]{\ensuremath\std[\underline{\sy}]}

\newcommand{\vnwi}[1][\idxC]{\ensuremath\underline{\vn}_{#1}}
\newcommand{\snwi}[1][]{\ensuremath\underline{\sn}_{#1}}

\newcommand{\sgnc}[1][]{\ensuremath\std[\underline{\sn}]}
\newcommand{\vdwi}[1][\idxC]{\ensuremath\underline{\vd}_{#1}}

% real valued
\newcommand{\vbwi}[1][]{\ensuremath\vb_{#1}}

\newcommand{\scwi}[1][\idxSa]{\ensuremath\sco_{#1}}
\newcommand{\vcwi}[1][]{\ensuremath\vc_{#1}}

\newcommand{\mPwi}[1][\idxC]{\ensuremath\mP_{#1}}
\newcommand{\defmPwi}[1][\idxC]{\ensuremath\mP_{#1} =\diag[\sP_{\idxC,1},\dotsc,\sP_{\idxC,2\nA}]}

\newcommand{\mPrewir}[1][\idxC]{\ensuremath\mPre_{#1}}
\newcommand{\mtPrewir}[1][\idxC]{\ensuremath\tilde{\mPre}_{#1}}
\newcommand{\sNormwi}[1][\idxC]{\ensuremath\sNorm_{#1}}

\newcommand{\mRwir}[1][\idxC]{\ensuremath\mR_{#1}}
\newcommand{\mRwirp}[1][\idxC]{\ensuremath\bar{\mR}_{#1}}
\newcommand{\mdRwir}[1][\idxC]{\ensuremath\boldsymbol{\Gamma}_{#1}}
\newcommand{\mLwir}[1][\idxC]{\ensuremath\mL_{#1}}
\newcommand{\mQwir}[1][\idxC]{\ensuremath\mQ_{#1}}

\newcommand{\mUwir}[1][\idxC]{\ensuremath\mU_{#1}}
\newcommand{\mSwir}[1][\idxC]{\ensuremath\mS_{#1}}
\newcommand{\mVwir}[1][\idxC]{\ensuremath\mV_{#1}}

\newcommand{\mUwire}[1][\idxC]{\ensuremath\hat{\mU}_{#1}}
\newcommand{\mSwire}[1][\idxC]{\ensuremath\hat{\mS}_{#1}}
\newcommand{\mVwire}[1][\idxC]{\ensuremath\hat{\mV}_{#1}}

\newcommand{\sPrecwir}[1][\idxC,\idxA]{\ensuremath\sPrec_{#1}}
\newcommand{\vPrecwir}[1][\idxC]{\ensuremath\vPrec_{#1}}
\newcommand{\mBPrewir}[1][\idxC]{\ensuremath\mBPre_{#1}}

\newcommand{\vxwir}[1][\idxC]{\ensuremath\vx_{#1}}
\newcommand{\vdwir}[1][\idxC]{\ensuremath\vd_{#1}}
\newcommand{\sdwir}[1][\idxC,\idxA]{\ensuremath\sd_{#1}}

\newcommand{\sywir}[1][\idxC,\idxA]{\ensuremath\sy_{#1}}
\newcommand{\vywir}[1][\idxC]{\ensuremath\vy_{#1}}

\newcommand{\snwir}[1][\idxC,\idxA]{\ensuremath\sn_{#1}}
\newcommand{\vnwir}[1][\idxC]{\ensuremath\vn_{#1}}
\newcommand{\mSywir}[1][\idxC]{\ensuremath\mSy_{#1}}
\newcommand{\mSywire}[1][\idxC]{\ensuremath\mSye_{#1}}

\newcommand{\mSywid}[1][\idxC]{\ensuremath{\mHd}_{#1}}
\newcommand{\sSywid}[1][\idxC]{\ensuremath{\sHd}_{#1}}
\newcommand{\vSywid}[1][\idxC]{\ensuremath{\vHd}_{#1}}

\newcommand{\mWwir}[1][\idxC]{\ensuremath\mW_{#1}}
\newcommand{\szwir}[1][\idxC,\idxA]{\ensuremath\sz_{#1}}
\newcommand{\vzwir}[1][\idxC]{\ensuremath\vz_{#1}}

% effective system model
\newcommand{\sneff}[1][\idxC,\idxA]{\ensuremath\tilde{\sn}_{#1}}
\newcommand{\vneff}[1][\idxC]{\ensuremath\tilde{\vn}_{#1}}

% Hypothesis and estimated values
\newcommand{\sdwih}[1][\idxC,\idxA]{\ensuremath\tilde{\sd}_{#1}}
\newcommand{\vdwih}[1][\idxC]{\ensuremath\tilde{\vd}_{#1}}

\newcommand{\sdwim}[1][\idxC,\idxA]{\ensuremath\bar{\sd}_{#1}} % mean estimated vector
\newcommand{\ssgm}[1][\idxC,\idxA]{\ensuremath\std[#1]} % mean estimated vector

\newcommand{\scwih}[1][\idxSa]{\ensuremath\tilde{\sco}_{#1}} % hypothesis code bit

\newcommand{\sdwie}[1][\idxC,\idxA]{\ensuremath\hat{\sd}_{#1}}
\newcommand{\vdwie}[1][\idxC]{\ensuremath\hat{\vd}_{#1}}

\newcommand{\scwie}[1][\idxSa]{\ensuremath\hat{\sco}_{#1}} % Estimated code bit
\newcommand{\vcwie}[1][]{\ensuremath\hat{\vc}_{#1}} % Estimated code bit vector

\newcommand{\vbewi}[1][]{\ensuremath\vbe_{#1}}

%%% Local Variables: 
%%% mode: plain-tex
%%% TeX-master: "../diss"
%%% End: 

%
%%%%%%%%%%%%%% Page style and headers %%%%%%%%%%%%%%%
\pagestyle{fancyplain}
% 148 × 210
\setlength{\textwidth}{11.5cm}
\setlength{\textheight}{17.5cm}
\setlength{\headheight}{3ex}
\setlength{\topmargin}{-12.275mm}
\addtolength{\topmargin}{-3ex}
\setlength{\evensidemargin}{-3.4mm}
\setlength{\oddsidemargin}{-14.4mm}
%
\addtolength{\headwidth}{\marginparsep}
\setlength{\headwidth}{11.5cm}
%
% Base Header definition
\renewcommand{\chaptermark}[1]
               {\markboth{ #1}{}}
\renewcommand{\sectionmark}[1]
               {\markright{\thesection\ #1}}
\lhead[\fancyplain{}{\thepage}]%
      {\fancyplain{}{\footnotesize\rightmark}}
\rhead[\fancyplain{}{\footnotesize\leftmark}]%
      {\fancyplain{}{\thepage}}
\cfoot{}
%
%########################################################################
%########### New TeX commands and Custom Stuff ##########################
%########################################################################
% Definition of the german date
%
\makeatletter
\def\dategerman{\def\today{\month@german
    \space\number\year}}
\makeatother
%
% Phantom Section for formatting
\providecommand*{\phantomsection}{}
%
% A robust uppercase function to convert the first letter of every
% word in a sentence to upper case -> Glossary
%
\makeatletter
\newcommand*{\myMakeUpperCase}{}
\DeclareRobustCommand\myMakeUpperCase[1]{%
  \def\@myuppercasewords{\myuppercase@i#1 \@nil}%
 {\@myuppercasewords}%
}%
\def\myuppercase@i#1 #2\@nil{%
  \xmakefirstuc{#1}%
  \ifx\\#2\\%
  \else
    \@ReturnAfterFi{%
      \space
      \myuppercase@i#2\@nil
    }%
  \fi}%
\long\def\@ReturnAfterFi#1\fi{\fi#1}%
\makeatother

% ##########  Load tikZ and define styles  ##########
%% Define your own pgfplot and tikz default styles
% This file defines TikZ styles and options for exams! It is also used be the ant_exam class!
\usepgfplotslibrary{groupplots}
\usetikzlibrary{arrows,arrows.meta,automata,backgrounds,calc,shapes.misc,shapes.symbols,shapes.geometric,decorations.markings,decorations.pathmorphing,decorations.shapes,matrix,positioning,shadows,svg.path,plotmarks,spy,pgfplots.groupplots,fit,patterns}
\pgfplotsset{compat=newest}

\tikzstyle{line} = [draw, -latex']
\tikzset{
	conn/.style={circle, fill=black, inner sep=.1pt, draw, minimum height = 1mm},
	box/.style={rectangle, fill=white, inner sep=7pt, very thick, draw, drop shadow={fill=white, xshift=0mm, yshift=-0mm}},
	kreis/.style={circle, fill=white, inner sep=7pt, very thick, draw,drop shadow={fill=white, xshift=0mm, yshift=-0mm}},
	se/.style={-stealth, thick},
	plus/.style={oplus,circle, fill=white, inner sep=.5pt, thick, draw, minimum height=3.5mm},
	times/.style={otimes,circle, inner sep=.5pt, thick, draw, minimum height=3.5mm},
	oplus/.style={path picture={\draw[black](path picture bounding box.south) -- (path picture bounding box.north) (path picture bounding box.west) -- (path picture bounding box.east);}},
	otimes/.style={path picture={\draw[black](path picture bounding box.south west) -- (path picture bounding box.north east) (path picture bounding box.south east) -- (path picture bounding box.north west);}},
	antenna/.style={isosceles triangle,fill=black,shape border rotate=-90, inner sep=2pt},
  	invisible/.style={opacity=0},
  	visible on/.style={alt=#1{}{invisible}},
  	alt/.code args={<#1>#2#3}{%
    	\alt<#1>{\pgfkeysalso{#2}}{\pgfkeysalso{#3}} % \pgfkeysalso doesn't change the path
  	},
	}

\tikzstyle{pluscircle} = [draw, circle, path picture={\draw (path picture bounding box.south) -- (path picture bounding box.north) (path picture bounding box.west) -- (path picture bounding box.east);}]

\tikzstyle{block} = [draw, rectangle, minimum height=2em, minimum width=4em, align=center, inner sep=2pt]
\tikzstyle{backgroundblock} = [minimum height=2em, minimum width=6em, align=center,fill=gray!30, inner sep=4pt]

\tikzstyle{input} = [coordinate]
\tikzstyle{output} = [coordinate]


% Define the ORCID LOGO
\usepackage{scalerel}
\usetikzlibrary{svg.path}
\definecolor{orcidlogocol}{HTML}{A6CE39}
\tikzset{
	orcidlogo/.pic={
		\fill[orcidlogocol] svg{M256,128c0,70.7-57.3,128-128,128C57.3,256,0,198.7,0,128C0,57.3,57.3,0,128,0C198.7,0,256,57.3,256,128z};
		\fill[white] svg{M86.3,186.2H70.9V79.1h15.4v48.4V186.2z}
		svg{M108.9,79.1h41.6c39.6,0,57,28.3,57,53.6c0,27.5-21.5,53.6-56.8,53.6h-41.8V79.1z M124.3,172.4h24.5c34.9,0,42.9-26.5,42.9-39.7c0-21.5-13.7-39.7-43.7-39.7h-23.7V172.4z}
		svg{M88.7,56.8c0,5.5-4.5,10.1-10.1,10.1c-5.6,0-10.1-4.6-10.1-10.1c0-5.6,4.5-10.1,10.1-10.1C84.2,46.7,88.7,51.3,88.7,56.8z};
	}
}

\newcommand\orcidicon[1]{\href{https://orcid.org/#1}{\mbox{\scalerel*{
				\begin{tikzpicture}[yscale=-1,transform shape]
				\pic{orcidlogo};
				\end{tikzpicture}
			}{U}}}} % original: |

% Official colors of the University of Bremen
\definecolor{UBBlueDark}{RGB}{28,53,107}
\definecolor{UBBlueMed}{RGB}{13,104,176}
\definecolor{UBBlueVeryLight}{RGB}{192,209,226} % this one's not official
\definecolor{UBBlueLight}{RGB}{114,179,223}
\definecolor{UBOrangeDark}{RGB}{247,167,3}
\definecolor{UBOrangeLight}{RGB}{255,232,177}
\definecolor{UBRedMed}{RGB}{213,17,48}
\definecolor{UBRedDark}{RGB}{135,39,70}

%%%%
% Carsten Style
%%%%
% A custom phantom function to manipulate the position of exponents in
% tikz plots
\def\myphantom#1#2{
  \setbox0 = \hbox{$#1$}
  \hbox to #2\wd0{\hfill}
}

% Insert tikz package for pictures and define library packages
\usetikzlibrary{arrows,calc,shapes.geometric,decorations.markings,decorations.pathmorphing,decorations.shapes,matrix,positioning,shadows,plotmarks,spy,pgfplots.groupplots,fit,patterns}
\pgfplotsset{compat=newest}

% Custom definitions to tweak standard plot formatting, plot groups
% and so on
\pgfplotsset{
  plot coordinates/math parser=false,
  log base 10 number format code/.code={$\displaystyle\ifthenelse{\equal{#1}{0.0}}{10^{\scriptscriptstyle\myphantom{\scriptscriptstyle-}{.3}\pgfmathprintnumber{#1}}\myphantom{\scriptscriptstyle-}{.7}}{10^{\scriptscriptstyle\pgfmathprintnumber{#1}}}$},
  every axis y label/.style={at={(ticklabel cs:0.5)},rotate=90,left=1mm,anchor=center},
  every axis x label/.style={at={(ticklabel cs:0.5)},below=1mm,anchor=center},
  every axis/.append style={
      footnotesize,
      width = 0.5\textwidth,
      legend style={font=\scriptsize,legend cell align = left,legend pos =north west},
      tick style={semithick},
  },
  every axis plot/.append style={thick},
  group style={
      columns= 2,
      horizontal sep=1.15cm,
      every plot/.style= {width=0.53\textwidth},
  },
  filter discard warning=false,
}

% Custom styles for block diagrams, especially times, plus,
% modulations in a node, etc.
\tikzset{
	common/.style={font=\small},
	box/.style={rectangle, fill=white, inner sep=7pt, very thick, draw,drop shadow={fill=black, xshift=1mm, yshift=-1mm}},
	vbox/.style={box,rotate=90,text centered},
	conn/.style={circle, fill=black, inner sep=.1pt, draw, minimum height = 1mm},
	oplus/.style={
          path picture={
            \draw[black](path picture bounding box.south) -- (path picture bounding box.north)
            (path picture bounding box.west) -- (path picture bounding box.east);
          }
        },
	otimes/.style={
          path picture={
            \draw[black](path picture bounding box.south west) -- (path picture bounding box.north east)
            (path picture bounding box.south east) -- (path picture
            bounding box.north west);
          }
        },
	4qam/.style={box,minimum height=8mm,minimum width=8mm,
          path picture={
            \draw[black,thin,-stealth](0,-3.1mm) -- (0,3.5mm);
            \draw[black,thin,-stealth] (-3.1mm,0) -- (3.5mm,0);
            \foreach \x in {-1.414213562373095mm,1.414213562373095mm} {
              \foreach \y in {-1.414213562373095mm,1.414213562373095mm} {
                \draw[fill=black,thin] (\x,\y) circle
                [radius=.1mm];
              }
            }
          }
        },
	16qam/.style={box,minimum height=8mm,minimum width=8mm,
          path picture={
            \draw[black,thin,-stealth](0,-3.1mm) -- (0,3.5mm);
            \draw[black,thin,-stealth] (-3.1mm,0) -- (3.5mm,0);
            \foreach \x in {-0.632455532033676mm,0.632455532033676mm,-1.897366596101028mm,1.897366596101028mm}{
              \foreach \y in {-0.632455532033676mm,0.632455532033676mm,-1.897366596101028mm,1.897366596101028mm}
              {
                \draw[fill=black,thin] (\x,\y) circle
                [radius=.1mm];
              }
            }
          }
        },
	64qam/.style={box,minimum height=8mm,minimum width=8mm,
          path picture={
            \draw[black,thin,-stealth](0,-3.1mm) -- (0,3.5mm);
            \draw[black,thin,-stealth] (-3.1mm,0) -- (3.5mm,0);
            \foreach \x in {-0.308606699924184mm,0.308606699924184mm,-0.925820099772551mm,0.925820099772551mm,-1.543033499620919mm,1.543033499620919mm,-2.160246899469286mm,2.160246899469286mm}{
              \foreach \y in {-0.308606699924184mm,0.308606699924184mm,-0.925820099772551mm,0.925820099772551mm,-1.543033499620919mm,1.543033499620919mm,-2.160246899469286mm,2.160246899469286mm}
              {
                \draw[fill=black,thin] (\x,\y) circle [radius=.1mm];
              }
            }
          }
        },
	plus/.style={oplus,circle, fill=white, inner sep=.5pt, thick, draw, minimum height=3.5mm},
	times/.style={otimes,circle, inner sep=.5pt, thick, draw, minimum height=3.5mm},
	antenna/.style={isosceles triangle,fill=black, shape border rotate=-90, inner sep=2pt},
	se/.style={-stealth, thick},
	pa/.style={-stealth,thick,double distance between line centers=2pt,
          decoration={markings,mark=at position .4 with {
              \draw (-2pt,-2pt) -- (2pt,2pt);\draw (0pt,-2pt) -- (4pt,2pt);
            }
          }
        },
	dots/.style={dash pattern=on 1pt off 4pt,very thick},
	wave/.style={-stealth,decorate,
          decoration={coil,aspect=0,amplitude=.2mm,segment length=1mm,pre=lineto,pre length=1mm,post=lineto, post length=1mm}
        },
	exit_inner/.style={color=black,solid},
	exit_combine/.style={color=black,dash pattern=on 1.5pt off 3pt on 1.5pt off 3pt, line width=1.5pt},
	exit_outer_bg/.style={color=white!50!black!50,dashed},
	exit_outer/.style={color=black,dashed},
}
% Declaration of layers in pgf pictures, e.g., to draw something
% behind a picture (also useful for beamer)
\pgfdeclarelayer{background}
\pgfsetlayers{background,main}


% ##########  Style for floating objects  ###########
\setcounter{bottomnumber}{2} % default 1
\renewcommand{\bottomfraction}{.5} % default .3
%
% ########## List of Symbols and Index  ###########
\newglossary{symbols}{sim}{sym}{List of Symbols}
\newglossary{index}{ind}{idx}{Index}
\makeglossaries

% ######### Custom glossary styles ##############
% TODO: Requires update beyond glossaries package <= 4.49
% Deprecated command \glossarystyle
% Custom glossary style to ensure upper case letters for acronyms in
% the list if lower case versions are used in the text

% TODO: Requires update beyond glossaries package <= 4.49
% Deprecated command \glossarystyle

\newglossarystyle{mylist}{
  \glossarystyle{list}%
  \renewcommand*{\glossaryentryfield}[5]{\item[\glstarget{##1}{##2}] \myMakeUpperCase{##3}\glspostdescription\space}%
  \renewcommand*{\glossarysubentryfield}[6]{\glstarget{##2}{\strut}##4\glspostdescription\space}%
}

% Custom style for the symbol list to control the dotted line
\newglossarystyle{symbolidx}{%
  \glossarystyle{sublistdotted}% base this style on the list style
  \setlength{\glslistdottedwidth}{8em}
}

% Custom index style to control formatting
\newglossarystyle{myidx}{
  \glossarystyle{index}%
  \renewcommand*{\glossaryentryfield}[5]{%
  \item\glstarget{##1}{##2}%
    \ifx\relax##4\relax
    \else
    \space(##4)%
    \fi
    \space ##3\glspostdescription \space ##5}%

  \renewcommand*{\glossarysubentryfield}[6]{%
    \ifcase##1\relax
    % level 0
    \item
    \or
    % level 1
    \subitem
    \else
    % all other levels
    \subsubitem
    \fi
    \glstarget{##2}{##3}%
    \ifx\relax##5\relax
    \else
    \space(##5)%
    \fi
    \space##4\glspostdescription\space ##6}%
}

% ########## todo notes definitions ##########

% Resolve incompatibility of tikzexternalize with todonotes [todonotes are drawn with tikz]
\usepackage{letltxmacro}
\LetLtxMacro{\oldmissingfigure}{\missingfigure}
\renewcommand{\missingfigure}[2][]{\tikzexternaldisable\oldmissingfigure[{#1}]{#2}\tikzexternalenable}
\LetLtxMacro{\oldtodo}{\todo}
\renewcommand{\todo}[2][]{\tikzexternaldisable\oldtodo[#1]{#2}\tikzexternalenable}

% Custom commands for todo notes to distinguish different tasks
\newcommand{\addref}{\todo[color=red!40]{Add reference.}}
\newcommand{\rewrite}[2][]{\todo[color=green!40,#1]{#2}}

% ######### Workarounds/Fixes ###############

% This is a workaround to include matplotlib figures with integrated images.
% https://tex.stackexchange.com/a/600827
\newcommand\inputpgf[2]{{
        \let\pgfimageWithoutPath\pgfimage
        \renewcommand{\pgfimage}[2][]{\pgfimageWithoutPath[##1]{#1/##2}}
        \let\includegraphicsWithoutPath\includegraphics
        \renewcommand{\includegraphics}[2][]{\includegraphicsWithoutPath[##1]{#1/##2}}
        \input{#1/#2}
}}

% Math minus sign unicode char not defined
\DeclareUnicodeCharacter{2212}{-}


% ########## Page Layout Penalties #########
\widowpenalty=500
\clubpenalty=500

% ##########  Paragraph indentation  ###########
\parindent1em
\parskip0pt

% ########## Glossary entries ########
\loadglsentries{include/glossary}

\begin{document}
%################## Figure and Table Names #####################
\renewcommand{\figurename}{{\small {\bf Figure}}}
\renewcommand{\tablename}{{\small {\bf Table}}}
%
\sloppy
\frontmatter
\pagenumbering{Roman}
\renewcommand{\arraystretch}{1,0}
% #########  Erste Seite zum Kopieren auf die vordere Umschlagseite:  ########
% #########  Zweite Seite = Erste Seite im Bericht:  ########
\thispagestyle{empty}
\vfill
\begin{LARGE}
  \centerline{%
  \begin{bf}
    \begin{tabular}{c}
      \parbox{\dimexpr\linewidth-0\fboxsep}{\centering
        \Berichtstitel
      }
    \end{tabular}
  \end{bf}
  } % End of \centerline
\end{LARGE}
\vspace*{1cm}
\begin{center}
    % TODO: English or German title page ???
    {\Large Dissertation} \\[1ex] zur Erlangung des akademischen Grades \\[1ex]
            {\em Doktor der Ingenieurwissenschaften (Dr.-Ing.)} \\[1ex]
    vorgelegt dem Fachbereich 1 (Physik/Elektrotechnik)\\[1ex]
    der Universit{\"a}t Bremen\\
      \vspace*{0,8cm}

    von   \vspace*{0,8cm}\\
    {\large \Autor}\\
  \vspace*{1,6cm}
  \begin{tabular}{ll}
Eingereicht am:                   & -tbd-\\
Tag des öffentlichen Kolloquiums: & -tbd-\\
Gutachter der Dissertation:       &  Prof.~Dr.-Ing.~Armin~Dekorsy\\
                                  &  Prof.~Dr.-Ing.~Examiner~Secunda \\
Weitere Prüfer:                   &  Prof.~Dr.-Ing.~Gut~Achter \\
                                  &  Prof.~Dr.-Ing.~Achter~Guter \\

\end{tabular}
  \vspace*{1cm}

  \pgfimage[width=3.87cm]{logos/UniHB_logo_2021}
  \vfill

  Bremen, \today

\end{center}
\vfill
\newpage
\thispagestyle{empty}
 \phantom{x}
\newpage
% EOF

\renewcommand{\arraystretch}{1,5}
\renewcommand{\chaptermark}[1]
{\markboth{\thechapter\ #1}{}}
\cleardoublepage
%% The thank you part
%% Preface not in table of contents with \chapter* (default)
\chapter*{Preface}
% First sentence statement
The presented dissertation emerged from my work as a researcher with the Department of Communications Engineering at the University of Bremen.
% Thanks to "Doktorvater"
First and foremost I would like to thank Armin Dekorsy for offering me the opportunity to pursue my doctoral degree at the Department of Communications Engineering.
His continuous support and professional guidance enabled and encouraged me to successfully publish my results in numerous international conferences as well as to discuss and exchange ideas among peers to grow my scientific expertise as well as to grow personally.

%Thanks for examiners and further profs
Furthermore, I would like to extend my gratitude to Peter Rost for taking the time to assess this dissertation and for his valuable remarks.
The same goes for Steffen Paul and Karl-Ludwig Krieger for spending their time to read through this work and partaking in the colloquium as further examiners.
% Thanks to Jondral for his networking support
I would like to thank Friedrich K. Jondral for encouraging and supporting me to start this journey to pursue my doctorate.
% Thanks for the moniez!
I would further like to thank the German Federal Ministry of Education and Research (BMBF) for funding
a large portion of this work as part of the projects HiFlecs, TACNET 4.0, and IRLG.

% Thanks to Carsten
I am indebted to Carsten Bockelmann for his guidance, broad wisdom and skillful support during these years that made this thesis possible.
The countless discussions as well as support and time that enabled me to bring new ideas from inception to publication were immeasurably valuable.
% Thanks to coworkers
My coworkers deserve praise for making this time with the ANT as fruitful and enjoyable as anyone can only wish for.

% Thanks for friends, family, AND girlfriend
This multi year effort would not be possible without my friends and family who were always there for me.
I am deeply grateful for their support.
Finally, I would like to express my deepest gratitude to my girlfriend who supported me in every imaginable way to follow through with my work.
\\~\\
Bremen, Month 20XX\\
\\
Johannes Demel

\cleardoublepage
\rhead[\fancyplain{}{\footnotesize Table of Contents}]%
{\fancyplain{}{\thepage}}
\lhead[\fancyplain{}{\thepage}]%
{\fancyplain{}{\footnotesize Table of Contents}}

\tableofcontents
\mainmatter
%
% -------------- Original header----------------
%
\lhead[\fancyplain{}{\thepage}]%
{\fancyplain{}{\footnotesize\rightmark}}
\rhead[\fancyplain{}{\footnotesize\leftmark}]%
{\fancyplain{}{\thepage}}
% -------------------
%
%
% ------------ Real text HERE ---------------
%
\chapter{Introduction}
\label{chap:introduction}
Index entries can be defined like shown here in the source code and
\glsadd{idx:mimo} glossary entries are simply used per \ac{mimo}.
\section{Structure}
\label{sec:structure}


\section{Notation}
\label{sec:notation}


% EOF
%%% Local Variables: 
%%% mode: latex
%%% TeX-master: "diss"
%%% End: 

% System model
% \chapter{Chap1}
\label{chap:1}
\section{Overview}
\label{sec:ch1overview}
\section{Chapter Summary}
\label{sec:ch1summary}

%%% Local Variables: 
%%% mode: latex
%%% TeX-master: "../diss"
%%% End: 

% % Information Theory
\chapter{Learning Based Channel Estimation for OFDM systems}
\label{cha:learning_ch_est}
\section{Overview}
\label{sec:ch2overview}
\section{Chapter Summary}
\label{sec:ch2summary}


%%% Local Variables: 
%%% mode: latex
%%% TeX-master: "../diss"
%%% End: 

% SISO chapter
\chapter{Quantization and Compression for Channel State Information}
\label{chap:3}
\section{Overview}
\label{sec:ch3overview}
\section{Chapter Summary}
\label{sec:ch3summary}


\begin{figure}[tbh!]
    \centering
    %% Creator: Matplotlib, PGF backend
%%
%% To include the figure in your LaTeX document, write
%%   \input{<filename>.pgf}
%%
%% Make sure the required packages are loaded in your preamble
%%   \usepackage{pgf}
%%
%% Figures using additional raster images can only be included by \input if
%% they are in the same directory as the main LaTeX file. For loading figures
%% from other directories you can use the `import` package
%%   \usepackage{import}
%% and then include the figures with
%%   \import{<path to file>}{<filename>.pgf}
%%
%% Matplotlib used the following preamble
%%
\begingroup%
\makeatletter%
\begin{pgfpicture}%
\pgfpathrectangle{\pgfpointorigin}{\pgfqpoint{4.185894in}{2.370660in}}%
\pgfusepath{use as bounding box, clip}%
\begin{pgfscope}%
\pgfsetbuttcap%
\pgfsetmiterjoin%
\definecolor{currentfill}{rgb}{1.000000,1.000000,1.000000}%
\pgfsetfillcolor{currentfill}%
\pgfsetlinewidth{0.000000pt}%
\definecolor{currentstroke}{rgb}{1.000000,1.000000,1.000000}%
\pgfsetstrokecolor{currentstroke}%
\pgfsetdash{}{0pt}%
\pgfpathmoveto{\pgfqpoint{0.000000in}{0.000000in}}%
\pgfpathlineto{\pgfqpoint{4.185894in}{0.000000in}}%
\pgfpathlineto{\pgfqpoint{4.185894in}{2.370660in}}%
\pgfpathlineto{\pgfqpoint{0.000000in}{2.370660in}}%
\pgfpathclose%
\pgfusepath{fill}%
\end{pgfscope}%
\begin{pgfscope}%
\pgfsetbuttcap%
\pgfsetmiterjoin%
\definecolor{currentfill}{rgb}{1.000000,1.000000,1.000000}%
\pgfsetfillcolor{currentfill}%
\pgfsetlinewidth{0.000000pt}%
\definecolor{currentstroke}{rgb}{0.000000,0.000000,0.000000}%
\pgfsetstrokecolor{currentstroke}%
\pgfsetstrokeopacity{0.000000}%
\pgfsetdash{}{0pt}%
\pgfpathmoveto{\pgfqpoint{0.381946in}{0.350309in}}%
\pgfpathlineto{\pgfqpoint{4.185894in}{0.350309in}}%
\pgfpathlineto{\pgfqpoint{4.185894in}{2.332079in}}%
\pgfpathlineto{\pgfqpoint{0.381946in}{2.332079in}}%
\pgfpathclose%
\pgfusepath{fill}%
\end{pgfscope}%
\begin{pgfscope}%
\pgfpathrectangle{\pgfqpoint{0.381946in}{0.350309in}}{\pgfqpoint{3.803948in}{1.981770in}}%
\pgfusepath{clip}%
\pgfsetrectcap%
\pgfsetroundjoin%
\pgfsetlinewidth{0.803000pt}%
\definecolor{currentstroke}{rgb}{0.690196,0.690196,0.690196}%
\pgfsetstrokecolor{currentstroke}%
\pgfsetdash{}{0pt}%
\pgfpathmoveto{\pgfqpoint{0.554853in}{0.350309in}}%
\pgfpathlineto{\pgfqpoint{0.554853in}{2.332079in}}%
\pgfusepath{stroke}%
\end{pgfscope}%
\begin{pgfscope}%
\pgfsetbuttcap%
\pgfsetroundjoin%
\definecolor{currentfill}{rgb}{0.000000,0.000000,0.000000}%
\pgfsetfillcolor{currentfill}%
\pgfsetlinewidth{0.803000pt}%
\definecolor{currentstroke}{rgb}{0.000000,0.000000,0.000000}%
\pgfsetstrokecolor{currentstroke}%
\pgfsetdash{}{0pt}%
\pgfsys@defobject{currentmarker}{\pgfqpoint{0.000000in}{-0.048611in}}{\pgfqpoint{0.000000in}{0.000000in}}{%
\pgfpathmoveto{\pgfqpoint{0.000000in}{0.000000in}}%
\pgfpathlineto{\pgfqpoint{0.000000in}{-0.048611in}}%
\pgfusepath{stroke,fill}%
}%
\begin{pgfscope}%
\pgfsys@transformshift{0.554853in}{0.350309in}%
\pgfsys@useobject{currentmarker}{}%
\end{pgfscope}%
\end{pgfscope}%
\begin{pgfscope}%
\definecolor{textcolor}{rgb}{0.000000,0.000000,0.000000}%
\pgfsetstrokecolor{textcolor}%
\pgfsetfillcolor{textcolor}%
\pgftext[x=0.554853in,y=0.253087in,,top]{\color{textcolor}\rmfamily\fontsize{8.330000}{9.996000}\selectfont -20}%
\end{pgfscope}%
\begin{pgfscope}%
\pgfpathrectangle{\pgfqpoint{0.381946in}{0.350309in}}{\pgfqpoint{3.803948in}{1.981770in}}%
\pgfusepath{clip}%
\pgfsetrectcap%
\pgfsetroundjoin%
\pgfsetlinewidth{0.803000pt}%
\definecolor{currentstroke}{rgb}{0.690196,0.690196,0.690196}%
\pgfsetstrokecolor{currentstroke}%
\pgfsetdash{}{0pt}%
\pgfpathmoveto{\pgfqpoint{0.987120in}{0.350309in}}%
\pgfpathlineto{\pgfqpoint{0.987120in}{2.332079in}}%
\pgfusepath{stroke}%
\end{pgfscope}%
\begin{pgfscope}%
\pgfsetbuttcap%
\pgfsetroundjoin%
\definecolor{currentfill}{rgb}{0.000000,0.000000,0.000000}%
\pgfsetfillcolor{currentfill}%
\pgfsetlinewidth{0.803000pt}%
\definecolor{currentstroke}{rgb}{0.000000,0.000000,0.000000}%
\pgfsetstrokecolor{currentstroke}%
\pgfsetdash{}{0pt}%
\pgfsys@defobject{currentmarker}{\pgfqpoint{0.000000in}{-0.048611in}}{\pgfqpoint{0.000000in}{0.000000in}}{%
\pgfpathmoveto{\pgfqpoint{0.000000in}{0.000000in}}%
\pgfpathlineto{\pgfqpoint{0.000000in}{-0.048611in}}%
\pgfusepath{stroke,fill}%
}%
\begin{pgfscope}%
\pgfsys@transformshift{0.987120in}{0.350309in}%
\pgfsys@useobject{currentmarker}{}%
\end{pgfscope}%
\end{pgfscope}%
\begin{pgfscope}%
\definecolor{textcolor}{rgb}{0.000000,0.000000,0.000000}%
\pgfsetstrokecolor{textcolor}%
\pgfsetfillcolor{textcolor}%
\pgftext[x=0.987120in,y=0.253087in,,top]{\color{textcolor}\rmfamily\fontsize{8.330000}{9.996000}\selectfont -15}%
\end{pgfscope}%
\begin{pgfscope}%
\pgfpathrectangle{\pgfqpoint{0.381946in}{0.350309in}}{\pgfqpoint{3.803948in}{1.981770in}}%
\pgfusepath{clip}%
\pgfsetrectcap%
\pgfsetroundjoin%
\pgfsetlinewidth{0.803000pt}%
\definecolor{currentstroke}{rgb}{0.690196,0.690196,0.690196}%
\pgfsetstrokecolor{currentstroke}%
\pgfsetdash{}{0pt}%
\pgfpathmoveto{\pgfqpoint{1.419386in}{0.350309in}}%
\pgfpathlineto{\pgfqpoint{1.419386in}{2.332079in}}%
\pgfusepath{stroke}%
\end{pgfscope}%
\begin{pgfscope}%
\pgfsetbuttcap%
\pgfsetroundjoin%
\definecolor{currentfill}{rgb}{0.000000,0.000000,0.000000}%
\pgfsetfillcolor{currentfill}%
\pgfsetlinewidth{0.803000pt}%
\definecolor{currentstroke}{rgb}{0.000000,0.000000,0.000000}%
\pgfsetstrokecolor{currentstroke}%
\pgfsetdash{}{0pt}%
\pgfsys@defobject{currentmarker}{\pgfqpoint{0.000000in}{-0.048611in}}{\pgfqpoint{0.000000in}{0.000000in}}{%
\pgfpathmoveto{\pgfqpoint{0.000000in}{0.000000in}}%
\pgfpathlineto{\pgfqpoint{0.000000in}{-0.048611in}}%
\pgfusepath{stroke,fill}%
}%
\begin{pgfscope}%
\pgfsys@transformshift{1.419386in}{0.350309in}%
\pgfsys@useobject{currentmarker}{}%
\end{pgfscope}%
\end{pgfscope}%
\begin{pgfscope}%
\definecolor{textcolor}{rgb}{0.000000,0.000000,0.000000}%
\pgfsetstrokecolor{textcolor}%
\pgfsetfillcolor{textcolor}%
\pgftext[x=1.419386in,y=0.253087in,,top]{\color{textcolor}\rmfamily\fontsize{8.330000}{9.996000}\selectfont -10}%
\end{pgfscope}%
\begin{pgfscope}%
\pgfpathrectangle{\pgfqpoint{0.381946in}{0.350309in}}{\pgfqpoint{3.803948in}{1.981770in}}%
\pgfusepath{clip}%
\pgfsetrectcap%
\pgfsetroundjoin%
\pgfsetlinewidth{0.803000pt}%
\definecolor{currentstroke}{rgb}{0.690196,0.690196,0.690196}%
\pgfsetstrokecolor{currentstroke}%
\pgfsetdash{}{0pt}%
\pgfpathmoveto{\pgfqpoint{1.851653in}{0.350309in}}%
\pgfpathlineto{\pgfqpoint{1.851653in}{2.332079in}}%
\pgfusepath{stroke}%
\end{pgfscope}%
\begin{pgfscope}%
\pgfsetbuttcap%
\pgfsetroundjoin%
\definecolor{currentfill}{rgb}{0.000000,0.000000,0.000000}%
\pgfsetfillcolor{currentfill}%
\pgfsetlinewidth{0.803000pt}%
\definecolor{currentstroke}{rgb}{0.000000,0.000000,0.000000}%
\pgfsetstrokecolor{currentstroke}%
\pgfsetdash{}{0pt}%
\pgfsys@defobject{currentmarker}{\pgfqpoint{0.000000in}{-0.048611in}}{\pgfqpoint{0.000000in}{0.000000in}}{%
\pgfpathmoveto{\pgfqpoint{0.000000in}{0.000000in}}%
\pgfpathlineto{\pgfqpoint{0.000000in}{-0.048611in}}%
\pgfusepath{stroke,fill}%
}%
\begin{pgfscope}%
\pgfsys@transformshift{1.851653in}{0.350309in}%
\pgfsys@useobject{currentmarker}{}%
\end{pgfscope}%
\end{pgfscope}%
\begin{pgfscope}%
\definecolor{textcolor}{rgb}{0.000000,0.000000,0.000000}%
\pgfsetstrokecolor{textcolor}%
\pgfsetfillcolor{textcolor}%
\pgftext[x=1.851653in,y=0.253087in,,top]{\color{textcolor}\rmfamily\fontsize{8.330000}{9.996000}\selectfont -5}%
\end{pgfscope}%
\begin{pgfscope}%
\pgfpathrectangle{\pgfqpoint{0.381946in}{0.350309in}}{\pgfqpoint{3.803948in}{1.981770in}}%
\pgfusepath{clip}%
\pgfsetrectcap%
\pgfsetroundjoin%
\pgfsetlinewidth{0.803000pt}%
\definecolor{currentstroke}{rgb}{0.690196,0.690196,0.690196}%
\pgfsetstrokecolor{currentstroke}%
\pgfsetdash{}{0pt}%
\pgfpathmoveto{\pgfqpoint{2.283920in}{0.350309in}}%
\pgfpathlineto{\pgfqpoint{2.283920in}{2.332079in}}%
\pgfusepath{stroke}%
\end{pgfscope}%
\begin{pgfscope}%
\pgfsetbuttcap%
\pgfsetroundjoin%
\definecolor{currentfill}{rgb}{0.000000,0.000000,0.000000}%
\pgfsetfillcolor{currentfill}%
\pgfsetlinewidth{0.803000pt}%
\definecolor{currentstroke}{rgb}{0.000000,0.000000,0.000000}%
\pgfsetstrokecolor{currentstroke}%
\pgfsetdash{}{0pt}%
\pgfsys@defobject{currentmarker}{\pgfqpoint{0.000000in}{-0.048611in}}{\pgfqpoint{0.000000in}{0.000000in}}{%
\pgfpathmoveto{\pgfqpoint{0.000000in}{0.000000in}}%
\pgfpathlineto{\pgfqpoint{0.000000in}{-0.048611in}}%
\pgfusepath{stroke,fill}%
}%
\begin{pgfscope}%
\pgfsys@transformshift{2.283920in}{0.350309in}%
\pgfsys@useobject{currentmarker}{}%
\end{pgfscope}%
\end{pgfscope}%
\begin{pgfscope}%
\definecolor{textcolor}{rgb}{0.000000,0.000000,0.000000}%
\pgfsetstrokecolor{textcolor}%
\pgfsetfillcolor{textcolor}%
\pgftext[x=2.283920in,y=0.253087in,,top]{\color{textcolor}\rmfamily\fontsize{8.330000}{9.996000}\selectfont 0}%
\end{pgfscope}%
\begin{pgfscope}%
\pgfpathrectangle{\pgfqpoint{0.381946in}{0.350309in}}{\pgfqpoint{3.803948in}{1.981770in}}%
\pgfusepath{clip}%
\pgfsetrectcap%
\pgfsetroundjoin%
\pgfsetlinewidth{0.803000pt}%
\definecolor{currentstroke}{rgb}{0.690196,0.690196,0.690196}%
\pgfsetstrokecolor{currentstroke}%
\pgfsetdash{}{0pt}%
\pgfpathmoveto{\pgfqpoint{2.716187in}{0.350309in}}%
\pgfpathlineto{\pgfqpoint{2.716187in}{2.332079in}}%
\pgfusepath{stroke}%
\end{pgfscope}%
\begin{pgfscope}%
\pgfsetbuttcap%
\pgfsetroundjoin%
\definecolor{currentfill}{rgb}{0.000000,0.000000,0.000000}%
\pgfsetfillcolor{currentfill}%
\pgfsetlinewidth{0.803000pt}%
\definecolor{currentstroke}{rgb}{0.000000,0.000000,0.000000}%
\pgfsetstrokecolor{currentstroke}%
\pgfsetdash{}{0pt}%
\pgfsys@defobject{currentmarker}{\pgfqpoint{0.000000in}{-0.048611in}}{\pgfqpoint{0.000000in}{0.000000in}}{%
\pgfpathmoveto{\pgfqpoint{0.000000in}{0.000000in}}%
\pgfpathlineto{\pgfqpoint{0.000000in}{-0.048611in}}%
\pgfusepath{stroke,fill}%
}%
\begin{pgfscope}%
\pgfsys@transformshift{2.716187in}{0.350309in}%
\pgfsys@useobject{currentmarker}{}%
\end{pgfscope}%
\end{pgfscope}%
\begin{pgfscope}%
\definecolor{textcolor}{rgb}{0.000000,0.000000,0.000000}%
\pgfsetstrokecolor{textcolor}%
\pgfsetfillcolor{textcolor}%
\pgftext[x=2.716187in,y=0.253087in,,top]{\color{textcolor}\rmfamily\fontsize{8.330000}{9.996000}\selectfont 5}%
\end{pgfscope}%
\begin{pgfscope}%
\pgfpathrectangle{\pgfqpoint{0.381946in}{0.350309in}}{\pgfqpoint{3.803948in}{1.981770in}}%
\pgfusepath{clip}%
\pgfsetrectcap%
\pgfsetroundjoin%
\pgfsetlinewidth{0.803000pt}%
\definecolor{currentstroke}{rgb}{0.690196,0.690196,0.690196}%
\pgfsetstrokecolor{currentstroke}%
\pgfsetdash{}{0pt}%
\pgfpathmoveto{\pgfqpoint{3.148454in}{0.350309in}}%
\pgfpathlineto{\pgfqpoint{3.148454in}{2.332079in}}%
\pgfusepath{stroke}%
\end{pgfscope}%
\begin{pgfscope}%
\pgfsetbuttcap%
\pgfsetroundjoin%
\definecolor{currentfill}{rgb}{0.000000,0.000000,0.000000}%
\pgfsetfillcolor{currentfill}%
\pgfsetlinewidth{0.803000pt}%
\definecolor{currentstroke}{rgb}{0.000000,0.000000,0.000000}%
\pgfsetstrokecolor{currentstroke}%
\pgfsetdash{}{0pt}%
\pgfsys@defobject{currentmarker}{\pgfqpoint{0.000000in}{-0.048611in}}{\pgfqpoint{0.000000in}{0.000000in}}{%
\pgfpathmoveto{\pgfqpoint{0.000000in}{0.000000in}}%
\pgfpathlineto{\pgfqpoint{0.000000in}{-0.048611in}}%
\pgfusepath{stroke,fill}%
}%
\begin{pgfscope}%
\pgfsys@transformshift{3.148454in}{0.350309in}%
\pgfsys@useobject{currentmarker}{}%
\end{pgfscope}%
\end{pgfscope}%
\begin{pgfscope}%
\definecolor{textcolor}{rgb}{0.000000,0.000000,0.000000}%
\pgfsetstrokecolor{textcolor}%
\pgfsetfillcolor{textcolor}%
\pgftext[x=3.148454in,y=0.253087in,,top]{\color{textcolor}\rmfamily\fontsize{8.330000}{9.996000}\selectfont 10}%
\end{pgfscope}%
\begin{pgfscope}%
\pgfpathrectangle{\pgfqpoint{0.381946in}{0.350309in}}{\pgfqpoint{3.803948in}{1.981770in}}%
\pgfusepath{clip}%
\pgfsetrectcap%
\pgfsetroundjoin%
\pgfsetlinewidth{0.803000pt}%
\definecolor{currentstroke}{rgb}{0.690196,0.690196,0.690196}%
\pgfsetstrokecolor{currentstroke}%
\pgfsetdash{}{0pt}%
\pgfpathmoveto{\pgfqpoint{3.580720in}{0.350309in}}%
\pgfpathlineto{\pgfqpoint{3.580720in}{2.332079in}}%
\pgfusepath{stroke}%
\end{pgfscope}%
\begin{pgfscope}%
\pgfsetbuttcap%
\pgfsetroundjoin%
\definecolor{currentfill}{rgb}{0.000000,0.000000,0.000000}%
\pgfsetfillcolor{currentfill}%
\pgfsetlinewidth{0.803000pt}%
\definecolor{currentstroke}{rgb}{0.000000,0.000000,0.000000}%
\pgfsetstrokecolor{currentstroke}%
\pgfsetdash{}{0pt}%
\pgfsys@defobject{currentmarker}{\pgfqpoint{0.000000in}{-0.048611in}}{\pgfqpoint{0.000000in}{0.000000in}}{%
\pgfpathmoveto{\pgfqpoint{0.000000in}{0.000000in}}%
\pgfpathlineto{\pgfqpoint{0.000000in}{-0.048611in}}%
\pgfusepath{stroke,fill}%
}%
\begin{pgfscope}%
\pgfsys@transformshift{3.580720in}{0.350309in}%
\pgfsys@useobject{currentmarker}{}%
\end{pgfscope}%
\end{pgfscope}%
\begin{pgfscope}%
\definecolor{textcolor}{rgb}{0.000000,0.000000,0.000000}%
\pgfsetstrokecolor{textcolor}%
\pgfsetfillcolor{textcolor}%
\pgftext[x=3.580720in,y=0.253087in,,top]{\color{textcolor}\rmfamily\fontsize{8.330000}{9.996000}\selectfont 15}%
\end{pgfscope}%
\begin{pgfscope}%
\pgfpathrectangle{\pgfqpoint{0.381946in}{0.350309in}}{\pgfqpoint{3.803948in}{1.981770in}}%
\pgfusepath{clip}%
\pgfsetrectcap%
\pgfsetroundjoin%
\pgfsetlinewidth{0.803000pt}%
\definecolor{currentstroke}{rgb}{0.690196,0.690196,0.690196}%
\pgfsetstrokecolor{currentstroke}%
\pgfsetdash{}{0pt}%
\pgfpathmoveto{\pgfqpoint{4.012987in}{0.350309in}}%
\pgfpathlineto{\pgfqpoint{4.012987in}{2.332079in}}%
\pgfusepath{stroke}%
\end{pgfscope}%
\begin{pgfscope}%
\pgfsetbuttcap%
\pgfsetroundjoin%
\definecolor{currentfill}{rgb}{0.000000,0.000000,0.000000}%
\pgfsetfillcolor{currentfill}%
\pgfsetlinewidth{0.803000pt}%
\definecolor{currentstroke}{rgb}{0.000000,0.000000,0.000000}%
\pgfsetstrokecolor{currentstroke}%
\pgfsetdash{}{0pt}%
\pgfsys@defobject{currentmarker}{\pgfqpoint{0.000000in}{-0.048611in}}{\pgfqpoint{0.000000in}{0.000000in}}{%
\pgfpathmoveto{\pgfqpoint{0.000000in}{0.000000in}}%
\pgfpathlineto{\pgfqpoint{0.000000in}{-0.048611in}}%
\pgfusepath{stroke,fill}%
}%
\begin{pgfscope}%
\pgfsys@transformshift{4.012987in}{0.350309in}%
\pgfsys@useobject{currentmarker}{}%
\end{pgfscope}%
\end{pgfscope}%
\begin{pgfscope}%
\definecolor{textcolor}{rgb}{0.000000,0.000000,0.000000}%
\pgfsetstrokecolor{textcolor}%
\pgfsetfillcolor{textcolor}%
\pgftext[x=4.012987in,y=0.253087in,,top]{\color{textcolor}\rmfamily\fontsize{8.330000}{9.996000}\selectfont 20}%
\end{pgfscope}%
\begin{pgfscope}%
\definecolor{textcolor}{rgb}{0.000000,0.000000,0.000000}%
\pgfsetstrokecolor{textcolor}%
\pgfsetfillcolor{textcolor}%
\pgftext[x=2.283920in,y=0.098766in,,top]{\color{textcolor}\rmfamily\fontsize{8.330000}{9.996000}\selectfont LLR representative \(\displaystyle \tilde{c}\)}%
\end{pgfscope}%
\begin{pgfscope}%
\pgfpathrectangle{\pgfqpoint{0.381946in}{0.350309in}}{\pgfqpoint{3.803948in}{1.981770in}}%
\pgfusepath{clip}%
\pgfsetrectcap%
\pgfsetroundjoin%
\pgfsetlinewidth{0.803000pt}%
\definecolor{currentstroke}{rgb}{0.690196,0.690196,0.690196}%
\pgfsetstrokecolor{currentstroke}%
\pgfsetdash{}{0pt}%
\pgfpathmoveto{\pgfqpoint{0.381946in}{0.350309in}}%
\pgfpathlineto{\pgfqpoint{4.185894in}{0.350309in}}%
\pgfusepath{stroke}%
\end{pgfscope}%
\begin{pgfscope}%
\pgfsetbuttcap%
\pgfsetroundjoin%
\definecolor{currentfill}{rgb}{0.000000,0.000000,0.000000}%
\pgfsetfillcolor{currentfill}%
\pgfsetlinewidth{0.803000pt}%
\definecolor{currentstroke}{rgb}{0.000000,0.000000,0.000000}%
\pgfsetstrokecolor{currentstroke}%
\pgfsetdash{}{0pt}%
\pgfsys@defobject{currentmarker}{\pgfqpoint{-0.048611in}{0.000000in}}{\pgfqpoint{0.000000in}{0.000000in}}{%
\pgfpathmoveto{\pgfqpoint{0.000000in}{0.000000in}}%
\pgfpathlineto{\pgfqpoint{-0.048611in}{0.000000in}}%
\pgfusepath{stroke,fill}%
}%
\begin{pgfscope}%
\pgfsys@transformshift{0.381946in}{0.350309in}%
\pgfsys@useobject{currentmarker}{}%
\end{pgfscope}%
\end{pgfscope}%
\begin{pgfscope}%
\definecolor{textcolor}{rgb}{0.000000,0.000000,0.000000}%
\pgfsetstrokecolor{textcolor}%
\pgfsetfillcolor{textcolor}%
\pgftext[x=0.186343in,y=0.311729in,left,base]{\color{textcolor}\rmfamily\fontsize{8.330000}{9.996000}\selectfont -5}%
\end{pgfscope}%
\begin{pgfscope}%
\pgfpathrectangle{\pgfqpoint{0.381946in}{0.350309in}}{\pgfqpoint{3.803948in}{1.981770in}}%
\pgfusepath{clip}%
\pgfsetrectcap%
\pgfsetroundjoin%
\pgfsetlinewidth{0.803000pt}%
\definecolor{currentstroke}{rgb}{0.690196,0.690196,0.690196}%
\pgfsetstrokecolor{currentstroke}%
\pgfsetdash{}{0pt}%
\pgfpathmoveto{\pgfqpoint{0.381946in}{0.845751in}}%
\pgfpathlineto{\pgfqpoint{4.185894in}{0.845751in}}%
\pgfusepath{stroke}%
\end{pgfscope}%
\begin{pgfscope}%
\pgfsetbuttcap%
\pgfsetroundjoin%
\definecolor{currentfill}{rgb}{0.000000,0.000000,0.000000}%
\pgfsetfillcolor{currentfill}%
\pgfsetlinewidth{0.803000pt}%
\definecolor{currentstroke}{rgb}{0.000000,0.000000,0.000000}%
\pgfsetstrokecolor{currentstroke}%
\pgfsetdash{}{0pt}%
\pgfsys@defobject{currentmarker}{\pgfqpoint{-0.048611in}{0.000000in}}{\pgfqpoint{0.000000in}{0.000000in}}{%
\pgfpathmoveto{\pgfqpoint{0.000000in}{0.000000in}}%
\pgfpathlineto{\pgfqpoint{-0.048611in}{0.000000in}}%
\pgfusepath{stroke,fill}%
}%
\begin{pgfscope}%
\pgfsys@transformshift{0.381946in}{0.845751in}%
\pgfsys@useobject{currentmarker}{}%
\end{pgfscope}%
\end{pgfscope}%
\begin{pgfscope}%
\definecolor{textcolor}{rgb}{0.000000,0.000000,0.000000}%
\pgfsetstrokecolor{textcolor}%
\pgfsetfillcolor{textcolor}%
\pgftext[x=0.225695in,y=0.807171in,left,base]{\color{textcolor}\rmfamily\fontsize{8.330000}{9.996000}\selectfont 0}%
\end{pgfscope}%
\begin{pgfscope}%
\pgfpathrectangle{\pgfqpoint{0.381946in}{0.350309in}}{\pgfqpoint{3.803948in}{1.981770in}}%
\pgfusepath{clip}%
\pgfsetrectcap%
\pgfsetroundjoin%
\pgfsetlinewidth{0.803000pt}%
\definecolor{currentstroke}{rgb}{0.690196,0.690196,0.690196}%
\pgfsetstrokecolor{currentstroke}%
\pgfsetdash{}{0pt}%
\pgfpathmoveto{\pgfqpoint{0.381946in}{1.341194in}}%
\pgfpathlineto{\pgfqpoint{4.185894in}{1.341194in}}%
\pgfusepath{stroke}%
\end{pgfscope}%
\begin{pgfscope}%
\pgfsetbuttcap%
\pgfsetroundjoin%
\definecolor{currentfill}{rgb}{0.000000,0.000000,0.000000}%
\pgfsetfillcolor{currentfill}%
\pgfsetlinewidth{0.803000pt}%
\definecolor{currentstroke}{rgb}{0.000000,0.000000,0.000000}%
\pgfsetstrokecolor{currentstroke}%
\pgfsetdash{}{0pt}%
\pgfsys@defobject{currentmarker}{\pgfqpoint{-0.048611in}{0.000000in}}{\pgfqpoint{0.000000in}{0.000000in}}{%
\pgfpathmoveto{\pgfqpoint{0.000000in}{0.000000in}}%
\pgfpathlineto{\pgfqpoint{-0.048611in}{0.000000in}}%
\pgfusepath{stroke,fill}%
}%
\begin{pgfscope}%
\pgfsys@transformshift{0.381946in}{1.341194in}%
\pgfsys@useobject{currentmarker}{}%
\end{pgfscope}%
\end{pgfscope}%
\begin{pgfscope}%
\definecolor{textcolor}{rgb}{0.000000,0.000000,0.000000}%
\pgfsetstrokecolor{textcolor}%
\pgfsetfillcolor{textcolor}%
\pgftext[x=0.225695in,y=1.302614in,left,base]{\color{textcolor}\rmfamily\fontsize{8.330000}{9.996000}\selectfont 5}%
\end{pgfscope}%
\begin{pgfscope}%
\pgfpathrectangle{\pgfqpoint{0.381946in}{0.350309in}}{\pgfqpoint{3.803948in}{1.981770in}}%
\pgfusepath{clip}%
\pgfsetrectcap%
\pgfsetroundjoin%
\pgfsetlinewidth{0.803000pt}%
\definecolor{currentstroke}{rgb}{0.690196,0.690196,0.690196}%
\pgfsetstrokecolor{currentstroke}%
\pgfsetdash{}{0pt}%
\pgfpathmoveto{\pgfqpoint{0.381946in}{1.836637in}}%
\pgfpathlineto{\pgfqpoint{4.185894in}{1.836637in}}%
\pgfusepath{stroke}%
\end{pgfscope}%
\begin{pgfscope}%
\pgfsetbuttcap%
\pgfsetroundjoin%
\definecolor{currentfill}{rgb}{0.000000,0.000000,0.000000}%
\pgfsetfillcolor{currentfill}%
\pgfsetlinewidth{0.803000pt}%
\definecolor{currentstroke}{rgb}{0.000000,0.000000,0.000000}%
\pgfsetstrokecolor{currentstroke}%
\pgfsetdash{}{0pt}%
\pgfsys@defobject{currentmarker}{\pgfqpoint{-0.048611in}{0.000000in}}{\pgfqpoint{0.000000in}{0.000000in}}{%
\pgfpathmoveto{\pgfqpoint{0.000000in}{0.000000in}}%
\pgfpathlineto{\pgfqpoint{-0.048611in}{0.000000in}}%
\pgfusepath{stroke,fill}%
}%
\begin{pgfscope}%
\pgfsys@transformshift{0.381946in}{1.836637in}%
\pgfsys@useobject{currentmarker}{}%
\end{pgfscope}%
\end{pgfscope}%
\begin{pgfscope}%
\definecolor{textcolor}{rgb}{0.000000,0.000000,0.000000}%
\pgfsetstrokecolor{textcolor}%
\pgfsetfillcolor{textcolor}%
\pgftext[x=0.166667in,y=1.798056in,left,base]{\color{textcolor}\rmfamily\fontsize{8.330000}{9.996000}\selectfont 10}%
\end{pgfscope}%
\begin{pgfscope}%
\pgfpathrectangle{\pgfqpoint{0.381946in}{0.350309in}}{\pgfqpoint{3.803948in}{1.981770in}}%
\pgfusepath{clip}%
\pgfsetrectcap%
\pgfsetroundjoin%
\pgfsetlinewidth{0.803000pt}%
\definecolor{currentstroke}{rgb}{0.690196,0.690196,0.690196}%
\pgfsetstrokecolor{currentstroke}%
\pgfsetdash{}{0pt}%
\pgfpathmoveto{\pgfqpoint{0.381946in}{2.332079in}}%
\pgfpathlineto{\pgfqpoint{4.185894in}{2.332079in}}%
\pgfusepath{stroke}%
\end{pgfscope}%
\begin{pgfscope}%
\pgfsetbuttcap%
\pgfsetroundjoin%
\definecolor{currentfill}{rgb}{0.000000,0.000000,0.000000}%
\pgfsetfillcolor{currentfill}%
\pgfsetlinewidth{0.803000pt}%
\definecolor{currentstroke}{rgb}{0.000000,0.000000,0.000000}%
\pgfsetstrokecolor{currentstroke}%
\pgfsetdash{}{0pt}%
\pgfsys@defobject{currentmarker}{\pgfqpoint{-0.048611in}{0.000000in}}{\pgfqpoint{0.000000in}{0.000000in}}{%
\pgfpathmoveto{\pgfqpoint{0.000000in}{0.000000in}}%
\pgfpathlineto{\pgfqpoint{-0.048611in}{0.000000in}}%
\pgfusepath{stroke,fill}%
}%
\begin{pgfscope}%
\pgfsys@transformshift{0.381946in}{2.332079in}%
\pgfsys@useobject{currentmarker}{}%
\end{pgfscope}%
\end{pgfscope}%
\begin{pgfscope}%
\definecolor{textcolor}{rgb}{0.000000,0.000000,0.000000}%
\pgfsetstrokecolor{textcolor}%
\pgfsetfillcolor{textcolor}%
\pgftext[x=0.166667in,y=2.293499in,left,base]{\color{textcolor}\rmfamily\fontsize{8.330000}{9.996000}\selectfont 15}%
\end{pgfscope}%
\begin{pgfscope}%
\definecolor{textcolor}{rgb}{0.000000,0.000000,0.000000}%
\pgfsetstrokecolor{textcolor}%
\pgfsetfillcolor{textcolor}%
\pgftext[x=0.111111in,y=1.341194in,,bottom,rotate=90.000000]{\color{textcolor}\rmfamily\fontsize{8.330000}{9.996000}\selectfont SNR [dB]}%
\end{pgfscope}%
\begin{pgfscope}%
\pgfpathrectangle{\pgfqpoint{0.381946in}{0.350309in}}{\pgfqpoint{3.803948in}{1.981770in}}%
\pgfusepath{clip}%
\pgfsetrectcap%
\pgfsetroundjoin%
\pgfsetlinewidth{1.505625pt}%
\definecolor{currentstroke}{rgb}{0.121569,0.466667,0.705882}%
\pgfsetstrokecolor{currentstroke}%
\pgfsetdash{}{0pt}%
\pgfpathmoveto{\pgfqpoint{2.086342in}{0.336420in}}%
\pgfpathlineto{\pgfqpoint{2.083399in}{0.350309in}}%
\pgfpathlineto{\pgfqpoint{2.072724in}{0.399853in}}%
\pgfpathlineto{\pgfqpoint{2.060970in}{0.449397in}}%
\pgfpathlineto{\pgfqpoint{2.048512in}{0.498942in}}%
\pgfpathlineto{\pgfqpoint{2.035858in}{0.548486in}}%
\pgfpathlineto{\pgfqpoint{2.022487in}{0.598030in}}%
\pgfpathlineto{\pgfqpoint{2.008355in}{0.647574in}}%
\pgfpathlineto{\pgfqpoint{1.993416in}{0.697119in}}%
\pgfpathlineto{\pgfqpoint{1.977623in}{0.746663in}}%
\pgfpathlineto{\pgfqpoint{1.961719in}{0.796207in}}%
\pgfpathlineto{\pgfqpoint{1.944120in}{0.845751in}}%
\pgfpathlineto{\pgfqpoint{1.926430in}{0.895296in}}%
\pgfpathlineto{\pgfqpoint{1.906817in}{0.944840in}}%
\pgfpathlineto{\pgfqpoint{1.888209in}{0.994384in}}%
\pgfpathlineto{\pgfqpoint{1.866431in}{1.043929in}}%
\pgfpathlineto{\pgfqpoint{1.844636in}{1.093473in}}%
\pgfpathlineto{\pgfqpoint{1.821700in}{1.143017in}}%
\pgfpathlineto{\pgfqpoint{1.797564in}{1.192561in}}%
\pgfpathlineto{\pgfqpoint{1.770602in}{1.242106in}}%
\pgfpathlineto{\pgfqpoint{1.742057in}{1.291650in}}%
\pgfpathlineto{\pgfqpoint{1.713645in}{1.341194in}}%
\pgfpathlineto{\pgfqpoint{1.683713in}{1.390738in}}%
\pgfpathlineto{\pgfqpoint{1.650037in}{1.440283in}}%
\pgfpathlineto{\pgfqpoint{1.614286in}{1.489827in}}%
\pgfpathlineto{\pgfqpoint{1.578759in}{1.539371in}}%
\pgfpathlineto{\pgfqpoint{1.538498in}{1.588915in}}%
\pgfpathlineto{\pgfqpoint{1.492678in}{1.638460in}}%
\pgfpathlineto{\pgfqpoint{1.446564in}{1.688004in}}%
\pgfpathlineto{\pgfqpoint{1.397166in}{1.737548in}}%
\pgfpathlineto{\pgfqpoint{1.340509in}{1.787092in}}%
\pgfpathlineto{\pgfqpoint{1.279233in}{1.836637in}}%
\pgfpathlineto{\pgfqpoint{1.212811in}{1.886181in}}%
\pgfpathlineto{\pgfqpoint{1.140669in}{1.935725in}}%
\pgfpathlineto{\pgfqpoint{1.057302in}{1.985269in}}%
\pgfpathlineto{\pgfqpoint{0.971404in}{2.034814in}}%
\pgfpathlineto{\pgfqpoint{0.872037in}{2.084358in}}%
\pgfpathlineto{\pgfqpoint{0.763185in}{2.133902in}}%
\pgfpathlineto{\pgfqpoint{0.643884in}{2.183446in}}%
\pgfpathlineto{\pgfqpoint{0.513086in}{2.232991in}}%
\pgfpathlineto{\pgfqpoint{0.368057in}{2.280470in}}%
\pgfusepath{stroke}%
\end{pgfscope}%
\begin{pgfscope}%
\pgfpathrectangle{\pgfqpoint{0.381946in}{0.350309in}}{\pgfqpoint{3.803948in}{1.981770in}}%
\pgfusepath{clip}%
\pgfsetrectcap%
\pgfsetroundjoin%
\pgfsetlinewidth{1.505625pt}%
\definecolor{currentstroke}{rgb}{1.000000,0.498039,0.054902}%
\pgfsetstrokecolor{currentstroke}%
\pgfsetdash{}{0pt}%
\pgfpathmoveto{\pgfqpoint{2.163192in}{0.336420in}}%
\pgfpathlineto{\pgfqpoint{2.161483in}{0.350309in}}%
\pgfpathlineto{\pgfqpoint{2.155223in}{0.399853in}}%
\pgfpathlineto{\pgfqpoint{2.148433in}{0.449397in}}%
\pgfpathlineto{\pgfqpoint{2.141259in}{0.498942in}}%
\pgfpathlineto{\pgfqpoint{2.133907in}{0.548486in}}%
\pgfpathlineto{\pgfqpoint{2.126587in}{0.598030in}}%
\pgfpathlineto{\pgfqpoint{2.118456in}{0.647574in}}%
\pgfpathlineto{\pgfqpoint{2.110383in}{0.697119in}}%
\pgfpathlineto{\pgfqpoint{2.101398in}{0.746663in}}%
\pgfpathlineto{\pgfqpoint{2.092849in}{0.796207in}}%
\pgfpathlineto{\pgfqpoint{2.083577in}{0.845751in}}%
\pgfpathlineto{\pgfqpoint{2.074289in}{0.895296in}}%
\pgfpathlineto{\pgfqpoint{2.063496in}{0.944840in}}%
\pgfpathlineto{\pgfqpoint{2.054648in}{0.994384in}}%
\pgfpathlineto{\pgfqpoint{2.042959in}{1.043929in}}%
\pgfpathlineto{\pgfqpoint{2.033066in}{1.093473in}}%
\pgfpathlineto{\pgfqpoint{2.022033in}{1.143017in}}%
\pgfpathlineto{\pgfqpoint{2.010714in}{1.192561in}}%
\pgfpathlineto{\pgfqpoint{1.997280in}{1.242106in}}%
\pgfpathlineto{\pgfqpoint{1.984542in}{1.291650in}}%
\pgfpathlineto{\pgfqpoint{1.972387in}{1.341194in}}%
\pgfpathlineto{\pgfqpoint{1.960153in}{1.390738in}}%
\pgfpathlineto{\pgfqpoint{1.945311in}{1.440283in}}%
\pgfpathlineto{\pgfqpoint{1.931820in}{1.489827in}}%
\pgfpathlineto{\pgfqpoint{1.919647in}{1.539371in}}%
\pgfpathlineto{\pgfqpoint{1.904481in}{1.588915in}}%
\pgfpathlineto{\pgfqpoint{1.887749in}{1.638460in}}%
\pgfpathlineto{\pgfqpoint{1.872574in}{1.688004in}}%
\pgfpathlineto{\pgfqpoint{1.860312in}{1.737548in}}%
\pgfpathlineto{\pgfqpoint{1.841347in}{1.787092in}}%
\pgfpathlineto{\pgfqpoint{1.825374in}{1.836637in}}%
\pgfpathlineto{\pgfqpoint{1.806712in}{1.886181in}}%
\pgfpathlineto{\pgfqpoint{1.792106in}{1.935725in}}%
\pgfpathlineto{\pgfqpoint{1.771702in}{1.985269in}}%
\pgfpathlineto{\pgfqpoint{1.759580in}{2.034814in}}%
\pgfpathlineto{\pgfqpoint{1.736256in}{2.084358in}}%
\pgfpathlineto{\pgfqpoint{1.718515in}{2.133902in}}%
\pgfpathlineto{\pgfqpoint{1.702435in}{2.183446in}}%
\pgfpathlineto{\pgfqpoint{1.688610in}{2.232991in}}%
\pgfpathlineto{\pgfqpoint{1.664857in}{2.282535in}}%
\pgfpathlineto{\pgfqpoint{1.642205in}{2.332079in}}%
\pgfusepath{stroke}%
\end{pgfscope}%
\begin{pgfscope}%
\pgfpathrectangle{\pgfqpoint{0.381946in}{0.350309in}}{\pgfqpoint{3.803948in}{1.981770in}}%
\pgfusepath{clip}%
\pgfsetrectcap%
\pgfsetroundjoin%
\pgfsetlinewidth{1.505625pt}%
\definecolor{currentstroke}{rgb}{0.172549,0.627451,0.172549}%
\pgfsetstrokecolor{currentstroke}%
\pgfsetdash{}{0pt}%
\pgfpathmoveto{\pgfqpoint{2.216623in}{0.336420in}}%
\pgfpathlineto{\pgfqpoint{2.215700in}{0.350309in}}%
\pgfpathlineto{\pgfqpoint{2.212233in}{0.399853in}}%
\pgfpathlineto{\pgfqpoint{2.208580in}{0.449397in}}%
\pgfpathlineto{\pgfqpoint{2.204732in}{0.498942in}}%
\pgfpathlineto{\pgfqpoint{2.200679in}{0.548486in}}%
\pgfpathlineto{\pgfqpoint{2.196712in}{0.598030in}}%
\pgfpathlineto{\pgfqpoint{2.192238in}{0.647574in}}%
\pgfpathlineto{\pgfqpoint{2.188319in}{0.697119in}}%
\pgfpathlineto{\pgfqpoint{2.183420in}{0.746663in}}%
\pgfpathlineto{\pgfqpoint{2.178678in}{0.796207in}}%
\pgfpathlineto{\pgfqpoint{2.174285in}{0.845751in}}%
\pgfpathlineto{\pgfqpoint{2.169186in}{0.895296in}}%
\pgfpathlineto{\pgfqpoint{2.163361in}{0.944840in}}%
\pgfpathlineto{\pgfqpoint{2.159084in}{0.994384in}}%
\pgfpathlineto{\pgfqpoint{2.152833in}{1.043929in}}%
\pgfpathlineto{\pgfqpoint{2.148452in}{1.093473in}}%
\pgfpathlineto{\pgfqpoint{2.142544in}{1.143017in}}%
\pgfpathlineto{\pgfqpoint{2.137433in}{1.192561in}}%
\pgfpathlineto{\pgfqpoint{2.130464in}{1.242106in}}%
\pgfpathlineto{\pgfqpoint{2.124204in}{1.291650in}}%
\pgfpathlineto{\pgfqpoint{2.119090in}{1.341194in}}%
\pgfpathlineto{\pgfqpoint{2.112869in}{1.390738in}}%
\pgfpathlineto{\pgfqpoint{2.105464in}{1.440283in}}%
\pgfpathlineto{\pgfqpoint{2.099314in}{1.489827in}}%
\pgfpathlineto{\pgfqpoint{2.095054in}{1.539371in}}%
\pgfpathlineto{\pgfqpoint{2.087937in}{1.588915in}}%
\pgfpathlineto{\pgfqpoint{2.080955in}{1.638460in}}%
\pgfpathlineto{\pgfqpoint{2.074209in}{1.688004in}}%
\pgfpathlineto{\pgfqpoint{2.069986in}{1.737548in}}%
\pgfpathlineto{\pgfqpoint{2.059335in}{1.787092in}}%
\pgfpathlineto{\pgfqpoint{2.053902in}{1.836637in}}%
\pgfpathlineto{\pgfqpoint{2.046470in}{1.886181in}}%
\pgfpathlineto{\pgfqpoint{2.042756in}{1.935725in}}%
\pgfpathlineto{\pgfqpoint{2.033177in}{1.985269in}}%
\pgfpathlineto{\pgfqpoint{2.031893in}{2.034814in}}%
\pgfpathlineto{\pgfqpoint{2.019631in}{2.084358in}}%
\pgfpathlineto{\pgfqpoint{2.011961in}{2.133902in}}%
\pgfpathlineto{\pgfqpoint{2.010803in}{2.183446in}}%
\pgfpathlineto{\pgfqpoint{2.006013in}{2.232991in}}%
\pgfpathlineto{\pgfqpoint{1.996621in}{2.282535in}}%
\pgfpathlineto{\pgfqpoint{1.988091in}{2.332079in}}%
\pgfusepath{stroke}%
\end{pgfscope}%
\begin{pgfscope}%
\pgfpathrectangle{\pgfqpoint{0.381946in}{0.350309in}}{\pgfqpoint{3.803948in}{1.981770in}}%
\pgfusepath{clip}%
\pgfsetrectcap%
\pgfsetroundjoin%
\pgfsetlinewidth{1.505625pt}%
\definecolor{currentstroke}{rgb}{0.839216,0.152941,0.156863}%
\pgfsetstrokecolor{currentstroke}%
\pgfsetdash{}{0pt}%
\pgfpathmoveto{\pgfqpoint{2.262181in}{0.336420in}}%
\pgfpathlineto{\pgfqpoint{2.261887in}{0.350309in}}%
\pgfpathlineto{\pgfqpoint{2.260784in}{0.399853in}}%
\pgfpathlineto{\pgfqpoint{2.259624in}{0.449397in}}%
\pgfpathlineto{\pgfqpoint{2.258405in}{0.498942in}}%
\pgfpathlineto{\pgfqpoint{2.257122in}{0.548486in}}%
\pgfpathlineto{\pgfqpoint{2.255772in}{0.598030in}}%
\pgfpathlineto{\pgfqpoint{2.254353in}{0.647574in}}%
\pgfpathlineto{\pgfqpoint{2.253247in}{0.697119in}}%
\pgfpathlineto{\pgfqpoint{2.251709in}{0.746663in}}%
\pgfpathlineto{\pgfqpoint{2.250095in}{0.796207in}}%
\pgfpathlineto{\pgfqpoint{2.248887in}{0.845751in}}%
\pgfpathlineto{\pgfqpoint{2.247150in}{0.895296in}}%
\pgfpathlineto{\pgfqpoint{2.245331in}{0.944840in}}%
\pgfpathlineto{\pgfqpoint{2.244044in}{0.994384in}}%
\pgfpathlineto{\pgfqpoint{2.242108in}{1.043929in}}%
\pgfpathlineto{\pgfqpoint{2.240815in}{1.093473in}}%
\pgfpathlineto{\pgfqpoint{2.238778in}{1.143017in}}%
\pgfpathlineto{\pgfqpoint{2.237522in}{1.192561in}}%
\pgfpathlineto{\pgfqpoint{2.235414in}{1.242106in}}%
\pgfpathlineto{\pgfqpoint{2.233246in}{1.291650in}}%
\pgfpathlineto{\pgfqpoint{2.232130in}{1.341194in}}%
\pgfpathlineto{\pgfqpoint{2.229965in}{1.390738in}}%
\pgfpathlineto{\pgfqpoint{2.227779in}{1.440283in}}%
\pgfpathlineto{\pgfqpoint{2.225588in}{1.489827in}}%
\pgfpathlineto{\pgfqpoint{2.224980in}{1.539371in}}%
\pgfpathlineto{\pgfqpoint{2.222991in}{1.588915in}}%
\pgfpathlineto{\pgfqpoint{2.221087in}{1.638460in}}%
\pgfpathlineto{\pgfqpoint{2.219309in}{1.688004in}}%
\pgfpathlineto{\pgfqpoint{2.217702in}{1.737548in}}%
\pgfpathlineto{\pgfqpoint{2.213848in}{1.787092in}}%
\pgfpathlineto{\pgfqpoint{2.212519in}{1.836637in}}%
\pgfpathlineto{\pgfqpoint{2.211537in}{1.886181in}}%
\pgfpathlineto{\pgfqpoint{2.210993in}{1.935725in}}%
\pgfpathlineto{\pgfqpoint{2.207382in}{1.985269in}}%
\pgfpathlineto{\pgfqpoint{2.207690in}{2.034814in}}%
\pgfpathlineto{\pgfqpoint{2.204396in}{2.084358in}}%
\pgfpathlineto{\pgfqpoint{2.201154in}{2.133902in}}%
\pgfpathlineto{\pgfqpoint{2.203350in}{2.183446in}}%
\pgfpathlineto{\pgfqpoint{2.200934in}{2.232991in}}%
\pgfpathlineto{\pgfqpoint{2.198822in}{2.282535in}}%
\pgfpathlineto{\pgfqpoint{2.197114in}{2.332079in}}%
\pgfusepath{stroke}%
\end{pgfscope}%
\begin{pgfscope}%
\pgfpathrectangle{\pgfqpoint{0.381946in}{0.350309in}}{\pgfqpoint{3.803948in}{1.981770in}}%
\pgfusepath{clip}%
\pgfsetrectcap%
\pgfsetroundjoin%
\pgfsetlinewidth{1.505625pt}%
\definecolor{currentstroke}{rgb}{0.580392,0.403922,0.741176}%
\pgfsetstrokecolor{currentstroke}%
\pgfsetdash{}{0pt}%
\pgfpathmoveto{\pgfqpoint{2.305659in}{0.336420in}}%
\pgfpathlineto{\pgfqpoint{2.305953in}{0.350309in}}%
\pgfpathlineto{\pgfqpoint{2.307056in}{0.399853in}}%
\pgfpathlineto{\pgfqpoint{2.308216in}{0.449397in}}%
\pgfpathlineto{\pgfqpoint{2.309435in}{0.498942in}}%
\pgfpathlineto{\pgfqpoint{2.310718in}{0.548486in}}%
\pgfpathlineto{\pgfqpoint{2.312068in}{0.598030in}}%
\pgfpathlineto{\pgfqpoint{2.313487in}{0.647574in}}%
\pgfpathlineto{\pgfqpoint{2.314593in}{0.697119in}}%
\pgfpathlineto{\pgfqpoint{2.316130in}{0.746663in}}%
\pgfpathlineto{\pgfqpoint{2.317745in}{0.796207in}}%
\pgfpathlineto{\pgfqpoint{2.318953in}{0.845751in}}%
\pgfpathlineto{\pgfqpoint{2.320690in}{0.895296in}}%
\pgfpathlineto{\pgfqpoint{2.322509in}{0.944840in}}%
\pgfpathlineto{\pgfqpoint{2.323796in}{0.994384in}}%
\pgfpathlineto{\pgfqpoint{2.325732in}{1.043929in}}%
\pgfpathlineto{\pgfqpoint{2.327025in}{1.093473in}}%
\pgfpathlineto{\pgfqpoint{2.329062in}{1.143017in}}%
\pgfpathlineto{\pgfqpoint{2.330318in}{1.192561in}}%
\pgfpathlineto{\pgfqpoint{2.332426in}{1.242106in}}%
\pgfpathlineto{\pgfqpoint{2.334594in}{1.291650in}}%
\pgfpathlineto{\pgfqpoint{2.335710in}{1.341194in}}%
\pgfpathlineto{\pgfqpoint{2.337875in}{1.390738in}}%
\pgfpathlineto{\pgfqpoint{2.340061in}{1.440283in}}%
\pgfpathlineto{\pgfqpoint{2.342252in}{1.489827in}}%
\pgfpathlineto{\pgfqpoint{2.342860in}{1.539371in}}%
\pgfpathlineto{\pgfqpoint{2.344849in}{1.588915in}}%
\pgfpathlineto{\pgfqpoint{2.346753in}{1.638460in}}%
\pgfpathlineto{\pgfqpoint{2.348531in}{1.688004in}}%
\pgfpathlineto{\pgfqpoint{2.350138in}{1.737548in}}%
\pgfpathlineto{\pgfqpoint{2.353992in}{1.787092in}}%
\pgfpathlineto{\pgfqpoint{2.355321in}{1.836637in}}%
\pgfpathlineto{\pgfqpoint{2.356303in}{1.886181in}}%
\pgfpathlineto{\pgfqpoint{2.356847in}{1.935725in}}%
\pgfpathlineto{\pgfqpoint{2.360458in}{1.985269in}}%
\pgfpathlineto{\pgfqpoint{2.360150in}{2.034814in}}%
\pgfpathlineto{\pgfqpoint{2.363444in}{2.084358in}}%
\pgfpathlineto{\pgfqpoint{2.366686in}{2.133902in}}%
\pgfpathlineto{\pgfqpoint{2.364490in}{2.183446in}}%
\pgfpathlineto{\pgfqpoint{2.366906in}{2.232991in}}%
\pgfpathlineto{\pgfqpoint{2.369018in}{2.282535in}}%
\pgfpathlineto{\pgfqpoint{2.370726in}{2.332079in}}%
\pgfusepath{stroke}%
\end{pgfscope}%
\begin{pgfscope}%
\pgfpathrectangle{\pgfqpoint{0.381946in}{0.350309in}}{\pgfqpoint{3.803948in}{1.981770in}}%
\pgfusepath{clip}%
\pgfsetrectcap%
\pgfsetroundjoin%
\pgfsetlinewidth{1.505625pt}%
\definecolor{currentstroke}{rgb}{0.549020,0.337255,0.294118}%
\pgfsetstrokecolor{currentstroke}%
\pgfsetdash{}{0pt}%
\pgfpathmoveto{\pgfqpoint{2.351217in}{0.336420in}}%
\pgfpathlineto{\pgfqpoint{2.352140in}{0.350309in}}%
\pgfpathlineto{\pgfqpoint{2.355607in}{0.399853in}}%
\pgfpathlineto{\pgfqpoint{2.359260in}{0.449397in}}%
\pgfpathlineto{\pgfqpoint{2.363108in}{0.498942in}}%
\pgfpathlineto{\pgfqpoint{2.367161in}{0.548486in}}%
\pgfpathlineto{\pgfqpoint{2.371128in}{0.598030in}}%
\pgfpathlineto{\pgfqpoint{2.375602in}{0.647574in}}%
\pgfpathlineto{\pgfqpoint{2.379521in}{0.697119in}}%
\pgfpathlineto{\pgfqpoint{2.384420in}{0.746663in}}%
\pgfpathlineto{\pgfqpoint{2.389162in}{0.796207in}}%
\pgfpathlineto{\pgfqpoint{2.393555in}{0.845751in}}%
\pgfpathlineto{\pgfqpoint{2.398654in}{0.895296in}}%
\pgfpathlineto{\pgfqpoint{2.404479in}{0.944840in}}%
\pgfpathlineto{\pgfqpoint{2.408756in}{0.994384in}}%
\pgfpathlineto{\pgfqpoint{2.415007in}{1.043929in}}%
\pgfpathlineto{\pgfqpoint{2.419388in}{1.093473in}}%
\pgfpathlineto{\pgfqpoint{2.425296in}{1.143017in}}%
\pgfpathlineto{\pgfqpoint{2.430407in}{1.192561in}}%
\pgfpathlineto{\pgfqpoint{2.437376in}{1.242106in}}%
\pgfpathlineto{\pgfqpoint{2.443636in}{1.291650in}}%
\pgfpathlineto{\pgfqpoint{2.448750in}{1.341194in}}%
\pgfpathlineto{\pgfqpoint{2.454971in}{1.390738in}}%
\pgfpathlineto{\pgfqpoint{2.462376in}{1.440283in}}%
\pgfpathlineto{\pgfqpoint{2.468526in}{1.489827in}}%
\pgfpathlineto{\pgfqpoint{2.472786in}{1.539371in}}%
\pgfpathlineto{\pgfqpoint{2.479903in}{1.588915in}}%
\pgfpathlineto{\pgfqpoint{2.486885in}{1.638460in}}%
\pgfpathlineto{\pgfqpoint{2.493631in}{1.688004in}}%
\pgfpathlineto{\pgfqpoint{2.497854in}{1.737548in}}%
\pgfpathlineto{\pgfqpoint{2.508505in}{1.787092in}}%
\pgfpathlineto{\pgfqpoint{2.513938in}{1.836637in}}%
\pgfpathlineto{\pgfqpoint{2.521370in}{1.886181in}}%
\pgfpathlineto{\pgfqpoint{2.525084in}{1.935725in}}%
\pgfpathlineto{\pgfqpoint{2.534663in}{1.985269in}}%
\pgfpathlineto{\pgfqpoint{2.535947in}{2.034814in}}%
\pgfpathlineto{\pgfqpoint{2.548209in}{2.084358in}}%
\pgfpathlineto{\pgfqpoint{2.555879in}{2.133902in}}%
\pgfpathlineto{\pgfqpoint{2.557037in}{2.183446in}}%
\pgfpathlineto{\pgfqpoint{2.561827in}{2.232991in}}%
\pgfpathlineto{\pgfqpoint{2.571219in}{2.282535in}}%
\pgfpathlineto{\pgfqpoint{2.579749in}{2.332079in}}%
\pgfusepath{stroke}%
\end{pgfscope}%
\begin{pgfscope}%
\pgfpathrectangle{\pgfqpoint{0.381946in}{0.350309in}}{\pgfqpoint{3.803948in}{1.981770in}}%
\pgfusepath{clip}%
\pgfsetrectcap%
\pgfsetroundjoin%
\pgfsetlinewidth{1.505625pt}%
\definecolor{currentstroke}{rgb}{0.890196,0.466667,0.760784}%
\pgfsetstrokecolor{currentstroke}%
\pgfsetdash{}{0pt}%
\pgfpathmoveto{\pgfqpoint{2.404648in}{0.336420in}}%
\pgfpathlineto{\pgfqpoint{2.406357in}{0.350309in}}%
\pgfpathlineto{\pgfqpoint{2.412617in}{0.399853in}}%
\pgfpathlineto{\pgfqpoint{2.419407in}{0.449397in}}%
\pgfpathlineto{\pgfqpoint{2.426581in}{0.498942in}}%
\pgfpathlineto{\pgfqpoint{2.433933in}{0.548486in}}%
\pgfpathlineto{\pgfqpoint{2.441253in}{0.598030in}}%
\pgfpathlineto{\pgfqpoint{2.449384in}{0.647574in}}%
\pgfpathlineto{\pgfqpoint{2.457457in}{0.697119in}}%
\pgfpathlineto{\pgfqpoint{2.466442in}{0.746663in}}%
\pgfpathlineto{\pgfqpoint{2.474991in}{0.796207in}}%
\pgfpathlineto{\pgfqpoint{2.484263in}{0.845751in}}%
\pgfpathlineto{\pgfqpoint{2.493551in}{0.895296in}}%
\pgfpathlineto{\pgfqpoint{2.504344in}{0.944840in}}%
\pgfpathlineto{\pgfqpoint{2.513192in}{0.994384in}}%
\pgfpathlineto{\pgfqpoint{2.524881in}{1.043929in}}%
\pgfpathlineto{\pgfqpoint{2.534774in}{1.093473in}}%
\pgfpathlineto{\pgfqpoint{2.545807in}{1.143017in}}%
\pgfpathlineto{\pgfqpoint{2.557126in}{1.192561in}}%
\pgfpathlineto{\pgfqpoint{2.570560in}{1.242106in}}%
\pgfpathlineto{\pgfqpoint{2.583298in}{1.291650in}}%
\pgfpathlineto{\pgfqpoint{2.595453in}{1.341194in}}%
\pgfpathlineto{\pgfqpoint{2.607687in}{1.390738in}}%
\pgfpathlineto{\pgfqpoint{2.622529in}{1.440283in}}%
\pgfpathlineto{\pgfqpoint{2.636020in}{1.489827in}}%
\pgfpathlineto{\pgfqpoint{2.648193in}{1.539371in}}%
\pgfpathlineto{\pgfqpoint{2.663359in}{1.588915in}}%
\pgfpathlineto{\pgfqpoint{2.680091in}{1.638460in}}%
\pgfpathlineto{\pgfqpoint{2.695266in}{1.688004in}}%
\pgfpathlineto{\pgfqpoint{2.707528in}{1.737548in}}%
\pgfpathlineto{\pgfqpoint{2.726493in}{1.787092in}}%
\pgfpathlineto{\pgfqpoint{2.742466in}{1.836637in}}%
\pgfpathlineto{\pgfqpoint{2.761128in}{1.886181in}}%
\pgfpathlineto{\pgfqpoint{2.775734in}{1.935725in}}%
\pgfpathlineto{\pgfqpoint{2.796138in}{1.985269in}}%
\pgfpathlineto{\pgfqpoint{2.808260in}{2.034814in}}%
\pgfpathlineto{\pgfqpoint{2.831584in}{2.084358in}}%
\pgfpathlineto{\pgfqpoint{2.849325in}{2.133902in}}%
\pgfpathlineto{\pgfqpoint{2.865405in}{2.183446in}}%
\pgfpathlineto{\pgfqpoint{2.879230in}{2.232991in}}%
\pgfpathlineto{\pgfqpoint{2.902983in}{2.282535in}}%
\pgfpathlineto{\pgfqpoint{2.925635in}{2.332079in}}%
\pgfusepath{stroke}%
\end{pgfscope}%
\begin{pgfscope}%
\pgfpathrectangle{\pgfqpoint{0.381946in}{0.350309in}}{\pgfqpoint{3.803948in}{1.981770in}}%
\pgfusepath{clip}%
\pgfsetrectcap%
\pgfsetroundjoin%
\pgfsetlinewidth{1.505625pt}%
\definecolor{currentstroke}{rgb}{0.498039,0.498039,0.498039}%
\pgfsetstrokecolor{currentstroke}%
\pgfsetdash{}{0pt}%
\pgfpathmoveto{\pgfqpoint{2.481498in}{0.336420in}}%
\pgfpathlineto{\pgfqpoint{2.484441in}{0.350309in}}%
\pgfpathlineto{\pgfqpoint{2.495116in}{0.399853in}}%
\pgfpathlineto{\pgfqpoint{2.506870in}{0.449397in}}%
\pgfpathlineto{\pgfqpoint{2.519328in}{0.498942in}}%
\pgfpathlineto{\pgfqpoint{2.531982in}{0.548486in}}%
\pgfpathlineto{\pgfqpoint{2.545353in}{0.598030in}}%
\pgfpathlineto{\pgfqpoint{2.559485in}{0.647574in}}%
\pgfpathlineto{\pgfqpoint{2.574424in}{0.697119in}}%
\pgfpathlineto{\pgfqpoint{2.590217in}{0.746663in}}%
\pgfpathlineto{\pgfqpoint{2.606121in}{0.796207in}}%
\pgfpathlineto{\pgfqpoint{2.623720in}{0.845751in}}%
\pgfpathlineto{\pgfqpoint{2.641410in}{0.895296in}}%
\pgfpathlineto{\pgfqpoint{2.661023in}{0.944840in}}%
\pgfpathlineto{\pgfqpoint{2.679631in}{0.994384in}}%
\pgfpathlineto{\pgfqpoint{2.701409in}{1.043929in}}%
\pgfpathlineto{\pgfqpoint{2.723204in}{1.093473in}}%
\pgfpathlineto{\pgfqpoint{2.746140in}{1.143017in}}%
\pgfpathlineto{\pgfqpoint{2.770276in}{1.192561in}}%
\pgfpathlineto{\pgfqpoint{2.797238in}{1.242106in}}%
\pgfpathlineto{\pgfqpoint{2.825783in}{1.291650in}}%
\pgfpathlineto{\pgfqpoint{2.854195in}{1.341194in}}%
\pgfpathlineto{\pgfqpoint{2.884127in}{1.390738in}}%
\pgfpathlineto{\pgfqpoint{2.917803in}{1.440283in}}%
\pgfpathlineto{\pgfqpoint{2.953554in}{1.489827in}}%
\pgfpathlineto{\pgfqpoint{2.989081in}{1.539371in}}%
\pgfpathlineto{\pgfqpoint{3.029342in}{1.588915in}}%
\pgfpathlineto{\pgfqpoint{3.075162in}{1.638460in}}%
\pgfpathlineto{\pgfqpoint{3.121276in}{1.688004in}}%
\pgfpathlineto{\pgfqpoint{3.170674in}{1.737548in}}%
\pgfpathlineto{\pgfqpoint{3.227331in}{1.787092in}}%
\pgfpathlineto{\pgfqpoint{3.288607in}{1.836637in}}%
\pgfpathlineto{\pgfqpoint{3.355028in}{1.886181in}}%
\pgfpathlineto{\pgfqpoint{3.427171in}{1.935725in}}%
\pgfpathlineto{\pgfqpoint{3.510538in}{1.985269in}}%
\pgfpathlineto{\pgfqpoint{3.596436in}{2.034814in}}%
\pgfpathlineto{\pgfqpoint{3.695803in}{2.084358in}}%
\pgfpathlineto{\pgfqpoint{3.804655in}{2.133902in}}%
\pgfpathlineto{\pgfqpoint{3.923956in}{2.183446in}}%
\pgfpathlineto{\pgfqpoint{4.054754in}{2.232991in}}%
\pgfpathlineto{\pgfqpoint{4.199783in}{2.280470in}}%
\pgfusepath{stroke}%
\end{pgfscope}%
\begin{pgfscope}%
\pgfsetrectcap%
\pgfsetmiterjoin%
\pgfsetlinewidth{0.803000pt}%
\definecolor{currentstroke}{rgb}{0.000000,0.000000,0.000000}%
\pgfsetstrokecolor{currentstroke}%
\pgfsetdash{}{0pt}%
\pgfpathmoveto{\pgfqpoint{0.381946in}{0.350309in}}%
\pgfpathlineto{\pgfqpoint{0.381946in}{2.332079in}}%
\pgfusepath{stroke}%
\end{pgfscope}%
\begin{pgfscope}%
\pgfsetrectcap%
\pgfsetmiterjoin%
\pgfsetlinewidth{0.803000pt}%
\definecolor{currentstroke}{rgb}{0.000000,0.000000,0.000000}%
\pgfsetstrokecolor{currentstroke}%
\pgfsetdash{}{0pt}%
\pgfpathmoveto{\pgfqpoint{4.185894in}{0.350309in}}%
\pgfpathlineto{\pgfqpoint{4.185894in}{2.332079in}}%
\pgfusepath{stroke}%
\end{pgfscope}%
\begin{pgfscope}%
\pgfsetrectcap%
\pgfsetmiterjoin%
\pgfsetlinewidth{0.803000pt}%
\definecolor{currentstroke}{rgb}{0.000000,0.000000,0.000000}%
\pgfsetstrokecolor{currentstroke}%
\pgfsetdash{}{0pt}%
\pgfpathmoveto{\pgfqpoint{0.381946in}{0.350309in}}%
\pgfpathlineto{\pgfqpoint{4.185894in}{0.350309in}}%
\pgfusepath{stroke}%
\end{pgfscope}%
\begin{pgfscope}%
\pgfsetrectcap%
\pgfsetmiterjoin%
\pgfsetlinewidth{0.803000pt}%
\definecolor{currentstroke}{rgb}{0.000000,0.000000,0.000000}%
\pgfsetstrokecolor{currentstroke}%
\pgfsetdash{}{0pt}%
\pgfpathmoveto{\pgfqpoint{0.381946in}{2.332079in}}%
\pgfpathlineto{\pgfqpoint{4.185894in}{2.332079in}}%
\pgfusepath{stroke}%
\end{pgfscope}%
\end{pgfpicture}%
\makeatother%
\endgroup%

    \caption{\gls{SNR} dependent quantizer \gls{LLR} representatives for $\re$ / $\im$ \gls{QPSK} components with $I_q=8$, $N_Q=1024$.}
    \label{fig:cloudran:quantizer:llrs}
\end{figure}

This \reffig{fig:cloudran:quantizer:llrs} is an example of a TikZ plot.



\begin{figure}[tbh!]
    \centering
    \includeinkscape[width=\textwidth]{hier_multicarrier_sync}
    \caption{Internals of the \textit{XFDMSync} \textit{Multicarrier Sync} hierarchical flowgraph.}
    \label{fig:fg:sync}
\end{figure}

% In \ref{fig:fg:sync} you can see how one may include pictures that are modified in Inkscape. Try to include them this way to adjust fonts etc.

% \begin{figure}[htb]
%     \centering
%     \begin{tikzpicture}[auto,node distance=3cm and .9cm,>=latex']
\tikzstyle{block} = [draw, rectangle, minimum height=2em, minimum width=2em, align=center, inner sep=2pt]

\tikzstyle{input} = [coordinate]
\tikzstyle{output} = [coordinate]

\tikzset{fgadd/.style={path picture={ 
  \draw[black]
(path picture bounding box.east) -- (path picture bounding box.west) (path picture bounding box.south) -- (path picture bounding box.north);
}}
}

\node [input] (src0) {};
\node [input,below=of src0] (src1) {};

\node [block, right=of src0] (channel00) {$\matChannelTaps_{0,0}$};
\node [block, below=.1 of channel00] (channel01) {$\matChannelTaps_{0, 1}$};
\node [block, right=of src1] (channel11) {$\matChannelTaps_{1,1}$};
\node [block, above=.1 of channel11] (channel10) {$\matChannelTaps_{1,0}$};



\node [draw,circle,fgadd, right=of channel00] (addnoise0) {};
\node [draw,circle,fgadd, right=of channel11] (addnoise1) {};
\node [above=.4 of addnoise0] (noise0) {$\vecNoise_0$};
\node [above=.4 of addnoise1] (noise1) {$\vecNoise_1$};

\node [output, right=of addnoise0] (snk0) {};
\node [output, below=of snk0] (snk1) {};


\draw [->] (src0) -- node [above left]{$\vecBaseband_0$}  (channel00);
\draw [->] (src0) --  (channel01);
\draw [->] (src1) -- node [above left]{$\vecBaseband_1$}  (channel10);
\draw [->] (src1) --  (channel11);

\draw [->] (channel00) --  (addnoise0);
\draw [->] (channel10) --  (addnoise0);
\draw [->] (channel01) --  (addnoise1);
\draw [->] (channel11) --  (addnoise1);

\draw [->] (noise0) -- (addnoise0);
\draw [->] (noise1) -- (addnoise1);
\draw [->] (addnoise0) -- node [above right]{$\vecBasebandR_0$}  (snk0);
\draw [->] (addnoise1) -- node [above right]{$\vecBasebandR_1$}  (snk1);

\end{tikzpicture}
%     \caption{Time domain $2\times 2$ channel model example flowgraph}\label{fig:flowgraph:channel}
% \end{figure}

Finally, \reffig{fig:flowgraph:channel} is a simple TikZ picture.
However, keep in mind that the math labels are taken from your definition file. You want to stay in sync.%
% MIMO chapter
\chapter{Link Adaptation Algorithms and Enhanced Techniques Based on SINR Sequence Prediction}
\label{chap:chp_LA}
\section{Overview}
\label{sec:ch4overview}
\section{Chapter Summary}
\label{sec:ch4summary}
%%% Local Variables: 
%%% mode: latex
%%% TeX-master: "../diss"
%%% End:
% Multi-User chapter
% \chapter{Chap5}
\label{cha:5}
\section{Overview}
\label{sec:ch5overview}
\section{Chapter Summary}
\label{sec:ch5summary}

%%% Local Variables: 
%%% mode: latex
%%% TeX-master: "../diss"
%%% End: 

% % Summary
\chapter{Summary}
\label{chap:summary}


%%% Local Variables: 
%%% mode: latex
%%% TeX-master: "diss"
%%% End: 

% Appendix
\lhead[\fancyplain{}{\thepage}]%
{\fancyplain{}{\footnotesize Appendix \rightmark}}
\rhead[\fancyplain{}{\footnotesize Appendix \leftmark}]%
{\fancyplain{}{\thepage}}

\begin{appendix}
\chapter{First Appendix}
\label{cha:app-first-appendix}

\end{appendix}

%%% Local Variables: 
%%% mode: latex
%%% TeX-master: "diss"
%%% End: 

% ----------------------------------------------------------------------
\backmatter
\clearpage
%
% --------------- Glossaries ----------------
%
\lhead[\fancyplain{}{\thepage}]%
{\fancyplain{}{\footnotesize\rightmark}}
\rhead[\fancyplain{}{\footnotesize\leftmark}]%
{\fancyplain{}{\thepage}}
% Add all entries of different glossary types for test purposes... Remove next line for final version
\glsaddall[types={\acronymtype,symbols,index}]
{\small
  \printglossary[type=\acronymtype,style=mylist]
  \printglossary[type=symbols,style=symbolidx]}
%
% -------------- Bibliography ---------------
%
% TODO: To be removed, backwards compatibility for bibtex
% \bibliographystyle{include/bibliography_style}
% All bib entries for test purposes.... Remove next line for final version
\nocite{*}
%
\lhead[\fancyplain{}{\thepage}]%
{\fancyplain{}{\footnotesize Bibliography}}
\rhead[\fancyplain{}{\footnotesize Bibliography}]%
{\fancyplain{}{\thepage}}
\phantomsection
\addcontentsline{toc}{chapter}{Bibliography}
{\small
  % TODO: To be removed, backwards compatibility for bibtex
  % \bibliography{include/bibliography}
  \printbibliography
}
%
% ----------------- Index -------------------
%
\renewcommand{\glspostdescription}{}
{\small\twocolumn \printglossary[type=index,style=myidx]}
\end{document}
